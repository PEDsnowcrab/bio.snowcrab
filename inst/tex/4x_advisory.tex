
\documentclass{beamer}

\mode<presentation> {
	%\usetheme{Hannover}
	%\usetheme{AnnArbor}
	\usetheme{Boadilla}
	%\usecolortheme{dolphin}
	\usecolortheme{seagull}
	\usefonttheme{structuresmallcapsserif}
	\beamertemplatenavigationsymbolsempty % turn off navigation
	\hypersetup{pdfstartview={Fit}} % fits the presentation to the window when first displayed
}

\usepackage{subfig}
\usepackage{graphicx} % Allows including images
\usepackage{booktabs} % Allows the use of \toprule, \midrule and \bottomrule in tables

%\usepackage{default}
%\usepackage{fourier}
%\usepackage[english]{babel}															% English language/hyphenation
%\usepackage[protrusion=true,expansion=true]{microtype}				% Better typography
%\usepackage[toc,page]{appendix}
%\usepackage[utf8]{inputenc}
%\usepackage{csquotes}
%\usepackage{hyperref}
\usepackage{graphicx} % Allows including images
%\usepackage{graphics}
%\usepackage{booktabs} % Allows the use of \toprule, \midrule and \bottomrule in tables
%\usepackage{amsmath,amsfonts}				% Math packages
\numberwithin{equation}{section}		% Equationnumbering: section.eq#
\numberwithin{figure}{section}	   	% Figurenumbering: section.fig#
\numberwithin{table}{section}				% Tablenumbering: section.tab#



%----------------------------------------------------------------------------------------
% these need to be incremented each year .. <<<<<<<<<<<<<<<<<<<<<
\newcommand{\present}{2018}
\newcommand{\yr}{2017}
\newcommand{\yrminusone}{2016}
\newcommand{\yrminustwo}{2015}
\newcommand{\yrminusthree}{2014}
\newcommand{\yrminusfour}{2013}
\newcommand{\yrminusfive}{2012}
% these need to be incremented each year .. <<<<<<<<<<<<<<<<<<<<<


% ----------------------------------
\newcommand{\D}{.}  % to replace dots .. 
\newcommand{\h}{C:/} %home directory
\newcommand{\e}{bio.data/}
\newcommand{\es}{bio.data/bio.snowcrab/} %snow crab assessment base directory
\newcommand{\eb}{bathymetry/}
\newcommand{\ea}{bio.data/aegis/}
\newcommand{\Ay}{assessments/2017/}
\newcommand{\A}{assessments/}
\newcommand{\m}{output/maps/} %base folder for maps
\newcommand{\mb}{output/maps/survey/snowcrab/annual/bycatch/} %survey bycatch folder
\newcommand{\main}{C:/bio.data/bio.snowcrab/assessments/2018/presentations/4X/}

% ----------------------------------
\date{June 19, 2018}
\title[4X Snow Crab Advisory \present]{4X Snow Crab Advisory\\  Maritimes Region\\ \present } 
\author[Snow Crab Unit]{Ben Zisserson, Brent Cameron, Amy Glass, Jae Choi} 
\institute[DFO Science]{
	Canadian Department of Fisheries and Oceans \\ % Your institution for the title page
	Science Branch \\
	Population Ecology Division
	\medskip
	\textit{} % Your email address
}


% ----------------------------------
\begin{document}
	
% ----------------------------------
\begin{frame}
\titlepage % Print the title page as the first slide

\end{frame}


% ----------------------------------
\begin{frame}
\frametitle{Overview} % Table of contents slide
\tableofcontents % \section{} and \subsection{} 
\end{frame}

%----------------------------------------------------------------------------------------
%	PRESENTATION SLIDES
%----------------------------------------------------------------------------------------


\section{Fishery}

\subsection{Landings}
\begin{frame}
\frametitle{Annual Landings}
\begin{figure}
 \vspace*{-0.4cm}
\centering
 \subfloat{\includegraphics[width=0.70\textwidth]{\main annual\D landings.pdf}}\
\end{figure}
\end{frame}

%\begin{frame}
%\frametitle{Annual Effort}
%\begin{figure}
%
% \centerline{\includegraphics[width=0.70\textwidth]{\main annual\D effort.pdf}}
% \end{figure}
%\end{frame}

%%------------------------------------------------
%
\begin{frame}
\subsection{Catch Rates}
\frametitle{Annual Catch Rates}
\begin{figure}
 \centerline{\includegraphics[width=0.70\textwidth]{\main annual\D cpue\D jack.pdf}}

 \end{figure}
\end{frame}

%--------------------------------------------------------------

\begin{frame}
\frametitle{Logbook locations}
 \vspace*{-0.65cm}
\begin{figure}
	\centering
	\subfloat{\includegraphics[width=0.36\textwidth]{\main logbook\D effort\D 2010.png}}\
	\subfloat{\includegraphics[width=0.36\textwidth]{\main logbook\D effort\D 2011.png}}\
	\subfloat{\includegraphics[width=0.36\textwidth]{\main logbook\D effort\D 2012.png}}\
	\subfloat{\includegraphics[width=0.36\textwidth]{\main logbook\D effort\D 2013.png}}\
	
\end{figure}
\end{frame}

\begin{frame}
\vspace*{-0.65cm}
\frametitle{Logbook locations}
\begin{figure}
\centering
\subfloat{\includegraphics[width=0.36\textwidth]{\main logbook\D effort\D 2014.png}}\
\subfloat{\includegraphics[width=0.36\textwidth]{\main logbook\D effort\D 2015.png}}\
\subfloat{\includegraphics[width=0.36\textwidth]{\main logbook\D effort\D 2016.png}}\
\subfloat{\includegraphics[width=0.36\textwidth]{\main logbook\D effort\D 2017.png}}\

\end{figure}
\end{frame}

%%------------------------------------------------
\subsection{Activity}
\begin{frame}
\frametitle{Active Vessels}
\begin{figure}
%

 \centerline{\includegraphics[width=0.70\textwidth]{\main annual\D vessels.pdf}}

 \end{figure}
\end{frame}


%%------------------------------------------------

\begin{frame}
\frametitle{Monthly Landings}
\begin{figure}

 \centerline{\includegraphics[width=0.70\textwidth]{\main monthly\D landings.pdf}}
 \end{figure}
\end{frame}

%
\section{At-Sea Observers}
\subsection{Sampling Details}
%
\begin{frame}
\frametitle{Observer Sampling}
%\vspace*{-2cm}
\begin{figure}
	
\subfloat{\includegraphics[width=0.8\textwidth]{\main observersummary.pdf}}
\end{figure}
\end{frame}

%%------------------------------------------------
%
\subsection{Catch Composition}
%
\begin{frame}
\frametitle{Observed Catch Composition}

\begin{columns}
\begin{column}<+->{0.5\textwidth}
%
\begin{figure}
\centerline{\includegraphics[width=1.0\textwidth]{\main cc\D obs\D past.pdf}}
%
\end{figure}
\end{column}
%
\begin{column}{0.5\textwidth}
\begin{figure}
%

\centerline{\includegraphics[width=1.0\textwidth]{\main cc\D obs\D present.pdf}}
%
\end{figure}
%
\end{column}
\end{columns}
*Dashed black line is 95mm MLS
%
\end{frame}



%%------------------------------------------------

\section{Environmental Conditions}

\begin{frame}
\frametitle{Fishery Temperatures}
\begin{figure}
	
	\centerline{\includegraphics[width=0.7\textwidth]{\main monthly\D temps.pdf}}
	
\end{figure}

\begin{itemize}
\item Temperatures from temperature loggers in fishermen's traps. Should be coldest of the year.
\item Colour blocks are snow crab temperature preferences. Green is ideal, yellow is marginal, red is potentially lethal longterm.
\end{itemize}
\end{frame}

%-----------------------------------------------------------------
\begin{frame}
\frametitle{Survey Temperatures}
\begin{figure}

\centerline{\includegraphics[width=0.6\textwidth]{\h \es \Ay timeseries/survey/t.pdf}}

\end{figure}
%Annual variations in bottom temperature observed during the ENS snow crab survey. The horizontal line indicates the long-term median temperature within each subarea
\end{frame}
%
%%------------------------------------------------
%

\begin{frame}
\frametitle{Shelf Wide Bottom Temperatures}
\begin{figure}
	\centering
	\subfloat{\includegraphics[width=0.4\textwidth]{\main temperatures\D bottom\D 2004.png}}\
	\subfloat{\includegraphics[width=0.4\textwidth]{\main temperatures\D bottom\D 2009.png}}\
	\subfloat{\includegraphics[width=0.4\textwidth]{\main temperatures\D bottom\D 2013.png}}\
	\subfloat{\includegraphics[width=0.4\textwidth]{\main temperatures\D bottom\D 2016.png}}\
\end{figure}
\end{frame}

\begin{frame}
\frametitle{Shelf Wide Bottom Temperatures}
\begin{figure}
	\centering
	\subfloat{\includegraphics[width=0.85\textwidth]{\main temperatures\D bottom\D 2017.png}}\
\end{figure}
\end{frame}

%%------------------------------------------------
%
\section{Assessment}
\subsection{Survey}
%
\begin{frame}
\frametitle{Survey Catches}
\vspace*{-0.3cm}
\begin{figure}
	\centering
	\subfloat{\includegraphics[width=0.45\textwidth]{\main R0\D mass\D 2012.png}}\
	\subfloat{\includegraphics[width=0.45\textwidth]{\main R0\D mass\D 2015.png}}\
	\subfloat{\includegraphics[width=0.45\textwidth]{\main R0\D mass\D 2016.png}}\
	\subfloat{\includegraphics[width=0.45\textwidth]{\main R0\D mass\D 2017.png}}\
\end{figure}
\end{frame}



%%------------------------------------------------

\begin{frame}
\frametitle{Population}
\begin{figure}
\centerline{\includegraphics[width=0.55\textwidth]{\main hist\D cfa4x.pdf}}

\end{figure}
\end{frame}
%

%
\subsection{Biomass Estimation}
%
\begin{frame}
\frametitle{Biomass Modelling}
\begin{itemize}
%
%
\item A new biomass estimation methodology was introduced in 2016 and further refined in 2017. 
\item This approach relates habitat and abundance with environmental and ecosystem variables while also accounting for spatial and temporal variation. 
\item The 2016 approach gave biomass estimates that potentially fluctuated greatly between years.
\item 2017 biomass estimation methodology further simplified model inputs (removing some ecosystem parameters) and added localized temporal smoothing. As a result, fishable biomass estimates are less variable from year to year.
\item 2016 methodology was overly optimistic for 4X. We believe the current methodology is more realistic and reflective of the 4X biomass
\item 4X is inherently hard to assess: large area with limited snow crab habitat and high environmental (temperature) variability 

\end{itemize}
\end{frame}

%%------------------------------------------------
%
%
\begin{frame}
\frametitle{Modelled Fishable Biomass}
\begin{figure}
	\centering
	\includegraphics[width=0.6\textwidth]{\main biomass\D timeseries.png}\\ 
\end{figure}
\vspace*{-0.3cm}
\begin{itemize}
	\item Red line: Biomass Survey Index; Blue Line: Modelled biomass estimate 
	\item Pre-fishery 4X biomass estimate was 120 t for 2017/18, relative to 149 t for previous year 
	\item Pre-2004 estimates are unreliable as no survey data, driven exclusively by catch rate / fishery information	
\end{itemize}

\end{frame}


\begin{frame}
\frametitle{Modelled Fishing Mortality}
\begin{figure}
\centering
\includegraphics[width=0.7\textwidth]{\main fishingmortality\D timeseries.png}\\
\end{figure}
\end{frame}


%%------------------------------------------------
\subsection{Harvest Advice}

\begin{frame}
\frametitle{Precautionary Approach}
\begin{figure}
	\centering
	\includegraphics[width=0.7\textwidth]{\h \es \A common/hcr_ideal.pdf}\\ 
Harvest control rules for the SSE snow crab fishery.
\end{figure}
\begin{itemize}
	\item Harvest control rules for the SSE snow crab fishery.
	\item Suite of biological indicators inform decisions within "Target Harvest Area"
\end{itemize}
\end{frame}


%------------------------------------------------
\subsection{Fisheries Model}
\begin{frame}
\frametitle{Fishable Biomass -- Biomass dynamics}

\begin{figure}
	\centering
	\includegraphics[width=0.8\textwidth]{\main hcr\D default.png}\\ 
\end{figure}
\begin{itemize}
	\item Pre-2004 biomass estimates likely shift the zones (critical, cautious, healthy) to higher values 
	\item This would not change the "critical" status based on past assessments and all other indicators (environment, catch rates, historical abundance, etc )
\end{itemize}
\end{frame}

%%------------------------------------------------

\section{Summary}
\begin{frame}
\frametitle{Summary}
\begin{itemize}
\item Lowest catch rates in time series even with very focused fishery footprint
\item TAC 50\% caught, even with minor season extension
\item No appreciable internal recruitment to fishery
\item Though survey temperatures lower in 2017 than 2016, overall viable snowcrab habitat severely constricted
\item Spatial extent of survey catches mirroring fishery and viable habitat footprints 
\item Stock is in the "critical zone"
\item Viability of this fishery is uncertain
%\item If we consider the stations sampled as representative of the area, population is in the healthy zone.
%\item Little confidence in biomass estimates but trends seem to align across multiple sources
%\item Temperatures were lower in 2014 than 2013 and 2012
%\item Apparent pulse of small crab in the area.
%\item \emph{Given these uncertainties but the indication that things are on the up turn an increase in TAC of 20-50\% of the 2014 TAC would likely have a negative impact on the stock}

\end{itemize}
\end{frame}


\end{document}
