



%----------------------------------------------------------------------------------------
%	Presentation for snow crab RAP using beamer
% last edit: Feb 2017

% to reduce the size of the PDF:
% use ghostscript:
% gs -q -dSAFER -dNOPAUSE -sDEVICE=pdfwrite -sPDFSETTINGS=printer -sOutputFile="resdoc2006sc.pdf" resdoc2006sc.pdf 

% the key being sPDFSETTINGS: with options: default, screen, ebook, printer, preprint
% try also:  -sCompressPages=true
% -sDownsampleColorImages=true
% -sColorImageResolution=300
% -sGrayImageResolution=300
% -sMonoImageResolution=300
% final choice to get it under 8MB :

% gs -q -dSAFER -dNOPAUSE -sDEVICE=pdfwrite -dPDFSETTINGS=/printer -dColorImageResolution=150 -dMonoImageResolution=150 -dGrayImageResolution=150  -dCompatibilityLevel=1.4 -sOutputFile="resdoc2006sc.printer.pdf" resdoc2006sc.pdf  < /dev/null

%----------------------------------------------------------------------------------------


\documentclass{beamer}

\usepackage{default}

\mode<presentation> {
  %\usetheme{Hannover}
  %\usetheme{AnnArbor}
  \usetheme{Boadilla}
  %\usecolortheme{dolphin}
  \usecolortheme{seagull}
  %\usefonttheme{structuresmallcapsserif}
  \beamertemplatenavigationsymbolsempty % turn off navigation
  \hypersetup{pdfstartview={Fit}} % fits the presentation to the window when first displayed
}


\usepackage{fourier}
\usepackage[english]{babel}															% English language/hyphenation
\usepackage[protrusion=true,expansion=true]{microtype}				% Better typography

\usepackage[toc,page]{appendix}
\usepackage[utf8]{inputenc}
\usepackage{csquotes}
\usepackage{hyperref}

% linking options
%%% Equation and float numbering
\usepackage{amsmath,amsfonts}										% Math packages
\numberwithin{equation}{section}		% Equationnumbering: section.eq#
\numberwithin{figure}{section}		% Figurenumbering: section.fig#
\numberwithin{table}{section}				% Tablenumbering: section.tab#

\usepackage{graphicx} % Allows including images
\usepackage{graphics}
\usepackage{booktabs} % Allows the use of \toprule, \midrule and \bottomrule in tables


\newcommand{\D}{.}  % to replace dots .. 
\newcommand{\bd}{\string~/bio.data}   %  \string~ is a representation of the home directory 
\newcommand{\bds}{\bd/bio.snowcrab}
\newcommand{\bdsa}{\bds/assessments}
\newcommand{\bdsay}{\bdsa/2016}


% these need to be incremented each year ..
\newcommand{\yr}{2016}
\newcommand{\yrminusone}{2015}
\newcommand{\yrminustwo}{2014}
\newcommand{\yrminusthree}{2013}

\newcommand{\A}{assessments/}
\newcommand{\Ay}{assessments/\yr/}

\newcommand{\spsmaps}{\bds/R/maps/survey/snowcrab/annual}



%----------------------------------------------------------------------------------------
%	TITLE PAGE
%----------------------------------------------------------------------------------------

\title{Snow Crab Assessment \\
  Martimes Region \\
  \yr } 

% The short title appears at the bottom of every slide, the full title is only on the title page

\author{Brad Hubley, Ben Zisserson, Brent Cameron, Jae S. Choi} % Your name
\institute[DFO] % Your institution as it will appear on the bottom of every slide, may be shorthand to save space
{
  Department of Fisheries and Oceans \\ % Your institution for the title page
  Science Branch \\
  Population Ecology Division
  \medskip
  \textit{} % Your email address
}
\date{\today} % Date, can be changed to a custom date

\begin{document}
  
  
  \begin{frame}
    \titlepage % Print the title page as the first slide
  \end{frame}
  
  \begin{frame}
    \frametitle{Overview} % Table of contents slide, comment this block out to remove it
    \tableofcontents % Throughout your presentation, if you choose to use \section{} and \subsection{} commands, these will automatically be printed on this slide as an overview of your presentation
  \end{frame}
  
  %----------------------------------------------------------------------------------------
  %	PRESENTATION SLIDES
  %----------------------------------------------------------------------------------------
  
  
  
  \begin{frame}
    \frametitle{4VWX Snow Crab}
    \begin{columns}[T]
      \begin{column}{0.4\textwidth}
        \begin{itemize}
          \item Southern limit of Snow Crab extent
          \item Three commercial fishing areas
        \end{itemize}
      \end{column}
      
      \begin{column}{0.6\textwidth}
        \begin{centering}
          Scotian Shelf and CFAs
          \begin{figure}
            \includegraphics[width=\textwidth]{\bds/maps/Basemap.pdf}
          \end{figure}
        \end{centering}
      \end{column}
    \end{columns}
  \end{frame}
  
  

%--------------------------------------------------------------------------------
%
%\section{Survey}
%
%\begin{frame}
%\frametitle{Survey \yr}
%	\begin{columns}
%	\begin{column}{0.3\textwidth}
%	Survey:
%	\begin{itemize}
%	\begin{tiny}
%		\item Western component of 4x completed this year
%		\item More fishing days than normal due to poor weather conditions
%		\item Same vessel, captain and Science staff as 2014\\
%	\end{tiny}
%	\end{itemize}
%
%	Area Patterns:
%	\begin{itemize}
%	\begin{tiny}
%		\item N-ENS: Less crab pretty much everywhere in 2015, except inner Glace Bay Hole
%		\item S-ENS: Less crab inshore and on the slope edge in 2015
%		\item 4x: Slightly more in some locations, less in others. Despite suspect numbers from last year\\
%	\end{tiny}
%	\end{itemize}
%	\end{column}
%	
%	\begin{column}{0.7\textwidth}
%		\begin{figure}
%  		\includegraphics[width=\textwidth]{\bdsay/figures/Basemap_yearlydifference.pdf}
%  		\end{figure}
% 	\end{column}
% 	\end{columns}
%\end{frame}
%--------------------------------------------------------------------------------

\section{Interactions with Other Species}

\begin{frame}
\frametitle{Interactions with Other Species}

\begin{itemize}
	\item Competitors: Competition with some benthic fish and crabs, but few strong competitors
	\item Prey: Echinoderms, shrimp, crabs, worms, bivalves, sea stars
	\item Predators: Atlantic Halibut, Atlantic Wolfish, Skates, Longhorn Cculpin, Sea Raven, Atlantic Cod, White Hake, American Plaice, and Haddock
\end{itemize}
\end{frame}





%------------------------------------------------------------------------------
\subsection{Competition/ Prey}

\begin{frame}
\frametitle{Competition/Prey}
Trends in biomass for potential predators/prey (log10, no\textbackslash$km^2$) of Snow Crab on the Scotian Shelf, in the Snow Crab Survey
	\begin{columns}
	\begin{column}{0.25\textwidth}
	\begin{itemize}
	  \setlength\itemsep{2em}
		\item[] Jonah Crab 
		\item[] Lesser Toad Crab
		\item[] Northern Shrimp
	\end{itemize}
	\end{column}

	\begin{column}{0.25\textwidth}
 	\begin{figure}
    \includegraphics[width=\textwidth]{\spsmaps/ms.no.2511/{ms.no.2511.\yrminusone}.png}\\   
    \includegraphics[width=\textwidth]{\spsmaps/ms.no.2521/{ms.no.2521.\yrminusone}.png}\\   
    \includegraphics[width=\textwidth]{\spsmaps/ms.no.2211/{ms.no.2211.\yrminusone}.png}  
  	\end{figure}
  	\end{column}

  	\begin{column}{0.25\textwidth}
 	\begin{figure}
    \includegraphics[width=\textwidth]{\spsmaps/ms.no.2511/{ms.no.2511.\yr}.png}\\   
    \includegraphics[width=\textwidth]{\spsmaps/ms.no.2521/{ms.no.2521.\yr}.png}\\    
    \includegraphics[width=\textwidth]{\spsmaps/ms.no.2211/{ms.no.2211.\yr}.png}  
  	\end{figure}
  	\end{column}

	\begin{column}{0.25\textwidth}
 	\begin{figure}
  	\includegraphics[width=0.6\textwidth]{\bdsay/timeseries/survey/{ms.mass.2511}.pdf}\\   
    \includegraphics[width=0.6\textwidth]{\bdsay/timeseries/survey/{ms.mass.2521}.pdf}\\
    \includegraphics[width=0.6\textwidth]{\bdsay/timeseries/survey/{ms.mass.2211}.pdf}
	\end{figure}
  	\end{column}

  	\end{columns}
\end{frame}

%------------------------------------------------
\subsection{Predation}

\begin{frame}
\frametitle{Potential Predation}
Trends in biomass for potential predators (log10, no\textbackslash$km^2$) of Snow Crab on the Scotian Shelf, in the Snow Crab Survey
	\begin{columns}
	\begin{column}{0.25\textwidth}
	\begin{itemize}
	  \setlength\itemsep{2em}
		\item[] Halibut 
		\item[] Atlantic Cod 
		\item[] Thorny Skate 
	\end{itemize}
	\end{column}

	\begin{column}{0.25\textwidth}
 	\begin{figure}
    \includegraphics[width=\textwidth]{\spsmaps/ms.mass.30/{ms.mass.30.\yrminusone}.png}\\   
    \includegraphics[width=\textwidth]{\spsmaps/ms.mass.10/{ms.mass.10.\yrminusone}.png}\\   
    \includegraphics[width=\textwidth]{\spsmaps/ms.mass.201/{ms.mass.201.\yrminusone}.png}  
  	\end{figure}
  	\end{column}

  	\begin{column}{0.25\textwidth}
 	\begin{figure}
    \includegraphics[width=\textwidth]{\spsmaps/ms.mass.30/{ms.mass.30.\yr}.png}\\   
    \includegraphics[width=\textwidth]{\spsmaps/ms.mass.10/{ms.mass.10.\yr}.png}\\   
    \includegraphics[width=\textwidth]{\spsmaps/ms.mass.201/{ms.mass.201.\yr}.png}  
  	\end{figure}
  	\end{column}

	\begin{column}{0.25\textwidth}
 	\begin{figure}
  	\includegraphics[width=0.6\textwidth]{\bdsay/timeseries/survey/{ms.mass.30}.pdf}\\   
    \includegraphics[width=0.6\textwidth]{\bdsay/timeseries/survey/{ms.mass.10}.pdf}\\
    \includegraphics[width=0.6\textwidth]{\bdsay/timeseries/survey/{ms.mass.201}.pdf}
	\end{figure}
  	\end{column}

  	\end{columns}
\end{frame}
%------------------------------------------------

\section{Population Assessment}
\subsection{Recruitment}

\begin{frame}
\frametitle{Recruitment}
\begin{columns}
	\begin{column}{0.5\textwidth}
	Size-frequency histograms of carapace width of male Snow Crab. The vertical line represents the legal size (95 mm)
	\begin{itemize}
		\item S-ENS: Stable recruitment
		\item N-ENS: A gap in recruitment, should enter the fishery in 1-2 years
		\item 4x: Minimal internal recruitment for the foreseeable future  
	\end{itemize}
	\end{column}

	\begin{column}{0.5\textwidth}
	\begin{figure}
		\includegraphics[width=\textwidth]{\bdsay/figures/size.freq/survey/male.pdf}
	\end{figure}
	\end{column}
	\end{columns}
\end{frame}

%--------------------------------------------------------------
\subsection{Reproductive Potential}

\begin{frame}
\frametitle{Sex Ratios}
	\begin{center}
	Proportion of females in the mature population
	\end{center}
	
	\begin{columns}
	\begin{column}{0.3\textwidth}
	\begin{itemize}
	\begin{footnotesize}
		\item S-ENS: Stable, much higher inshore this year...highest since 2008
		\item N-ENS: Recovering from near zero
		\item 4x: Continued high proportion of males  
	\end{footnotesize}
	\end{itemize}
	\end{column}

	\begin{column}{0.35\textwidth}
	\begin{figure}
	\includegraphics[width=\textwidth]{\bds/R/maps/survey/snowcrab/annual/{sexratio.mat}/{sexratio.mat.2015}.png}
	\end{figure}
	\end{column}

	\begin{column}{0.35\textwidth}
	\begin{figure}
	 \includegraphics[width=\textwidth]{\bdsay/timeseries/survey/{sexratio.mat}.pdf}
	\end{figure}
	\end{column}
	\end{columns}
\end{frame}

%------------------------------------------------
\begin{frame}
\frametitle{Female Size-Frequency}
\begin{columns}
	\begin{column}{0.5\textwidth}
		\begin{itemize}
		\item Size-frequency histograms of carapace width of female Snow Crab
		\item Newly matured female crab are expected in all areas for the next 3-4 years
		\item Each newly matured female should support egg production for 3-5 years
		\end{itemize}
	\end{column}

	\begin{column}{0.5\textwidth}
	\begin{figure}
		\includegraphics[width=\textwidth]{\bdsay/figures/{size.freq}/survey/female.pdf}
	\end{figure}
	\end{column}
	\end{columns}
\end{frame}

%------------------------------------------------
\begin{frame}
\frametitle{Primiparous/Multiparous}
\begin{columns}
	\begin{column}{0.5\textwidth}
	\begin{center}
	Primiparous 
	\end{center}
		\begin{figure}
		\includegraphics[width=\textwidth]{\bds/R/maps/survey/snowcrab/annual/{totno.female.primiparous}/{totno.female.primiparous.2015}.png}
	\end{figure}
	\end{column}

	\begin{column}{0.5\textwidth}
	\begin{center}
	Multiparous 
	\end{center}
	\begin{figure}
		\includegraphics[width=\textwidth]{\bds/R/maps/survey/snowcrab/annual/{totno.female.multiparous}/{totno.female.multiparous.\yr}.png}	
	\end{figure}
	\end{column}
	\end{columns}
\end{frame}

%------------------------------------------------
\begin{frame}
\frametitle{Mature Females}
	\begin{center}
	Total number of females in the mature population. Reproductive potential peaked in 2007/8 and continues to remain low (except 4x in 2012).
	\end{center}
	\begin{columns}
    \begin{column}{0.5\textwidth}
      \begin{figure}
        \includegraphics[width=\textwidth]{\bds/R/maps/survey/snowcrab/annual/totno.female.mat/{totno.female.mat.\yr}.png}	
      \end{figure}
    \end{column}

  	\begin{column}{0.5\textwidth}
      \begin{figure}
        \includegraphics[width=0.7\textwidth]{\bdsay/timeseries/survey/{totno.female.mat}.pdf}
      \end{figure}
    \end{column}
  \end{columns}
\end{frame}

%------------------------------------------------
\subsection{Fishable Biomasss geometric means}

\begin{frame}
\frametitle{Fishable Biomass Index -- Geometric Mean}
\begin{columns}

\begin{column}{0.5\textwidth}
	\begin{center}
	Fishable Biomass from Annual Snow Crab Survey. Log10 (t\textbackslash$km^2$).\\ 
	\end{center}
	\begin{figure}
		\includegraphics[width=0.9\textwidth]{\bds/R/maps/survey/snowcrab/annual/R0.mass/{R0.mass.\yr}.png}
	\end{figure}
\end{column}


\begin{column}{0.5\textwidth}
	\begin{center}
	Time series of the Fishable Biomass from Annual Snow Crab Survey (Geometric Mean)
	\end{center}
\begin{figure}
    \centering
    \includegraphics[width=0.7\textwidth]{\bdsay/timeseries/survey/{R0.mass}.pdf}
 \end{figure}
\end{column}

\end{columns}
\end{frame}

%
%%------------------------------------------------

\begin{frame}
\frametitle{Fishable Biomass Index -- LBM}
\begin{columns}

\begin{column}{0.5\textwidth}
  \begin{center}
    Viable habitat probability for \yr
  \end{center}
  \begin{figure}
    \includegraphics[width=0.9\textwidth]{\bds/maps/snowcrab.large.males_presence_absence/snowcrab/annual/{snowcrab.large.males_presence_absence.mean.\yr}.png}
    \end{figure}
\end{column}

\begin{column}{0.5\textwidth}
  \begin{center}
    Estimated fishable abundance ln(t per km sq) 
  \end{center}
  \begin{figure}
    \includegraphics[width=0.9\textwidth]{\bds/maps/fishable.biomass/snowcrab/{prediction.abundance.mean.\yr}.png}
  \end{figure}
\end{column}


\end{columns}
\end{frame}

%------------------------------------------------

\begin{frame}
  \frametitle{Fishable Biomass Index -- LBM}
    
   \begin{columns} 
    \begin{column}{0.5\textwidth}
      \begin{center}
        Estimated fishable abundance ln(t per km sq) 
      \end{center}
      \begin{figure}
        \includegraphics[width=0.9\textwidth]{\bds/maps/fishable.biomass/snowcrab/{prediction.abundance.mean.\yr}.png}
      \end{figure}
    \end{column}
    
    \begin{column}{0.5\textwidth}
      \begin{center}
        Time series of the Area Expanded Fishable Biomass Density
      \end{center}
      \begin{figure}
        \centering
        \includegraphics[width=0.7\textwidth]{\bdsay/timeseries/interpolated/{snowcrab.biomass.index}.png}
      \end{figure}
    \end{column}
    \end{columns}
\end{frame}


%------------------------------------------------
\section{Fisheries Model}

\subsection{Fishable Biomass -- Biomass dynamics}

\begin{frame}
\frametitle{Fishable Biomass -- Biomass dynamics}

\begin{columns} 
\begin{column}{0.5\textwidth}
 \begin{itemize}
  \begin{tiny}
 
     \item Red: The fishable biomass index
     \item Blue: The posterior mean fishable biomass estimated from the logistic model
     \item Grey: The density distribution of posterior fishable biomass estimates darkest area being medians and the 95\% Credible Intervals (CI). 
     \end{tiny}
       
    \end{itemize}
\end{column}

\begin{column}{0.5\textwidth}
  \begin{figure}[ht]
    \centering
    \includegraphics[width=0.9\textwidth]{\bdsay/{biomass.timeseries}.png}
  \end{figure}
\end{column}
\end{columns} 

\end{frame}

%------------------------------------------------
\subsection{Fishing Mortality}
\begin{frame}
\frametitle{Fishing Mortality}
\begin{columns}
\begin{column}{0.4\textwidth}
Mortality factors are associated with:
\begin{itemize}
\begin{footnotesize}
	\item Predation: Long-term risk if increase in predators
	\item Food limitation: Likely stable 
	\item Competition: Likely stable\\
\end{footnotesize}
\end{itemize}

Area trends:
\begin{itemize}
\begin{footnotesize}
	\item N-ENS: Increasing over the past several years
	\item S-ENS: Slightly above 20\% harvest rate
	\item 4x: Peaking in 2005 and 2011/2012
\end{footnotesize}
\end{itemize}
\end{column}

\begin{column}{0.6\textwidth}
\begin{footnotesize}
Fishing mortality
\end{footnotesize}

\begin{itemize}
\begin{tiny}
	\item[] Grey Area: Posterior density distributions, with median with 90\% CI 
	\item[] Dark-dashed line: is the 20\% harvest rate
	\item[] Red line: estimated FMSY 	
\end{tiny}
\end{itemize}

\begin{figure}
    \includegraphics[width=0.6\textwidth]{\bdsay/{fishingmortality.timeseries}.png}
\end{figure}
\end{column}

\end{columns}
\end{frame}

%
%%------------------------------------------------
%\section{Business}
%\subsection{Meetings / Document Schedule}
%
%\begin{frame}
%
%\frametitle{Meetings Schedule}
%
% \vspace*{-0.5cm}
%\begin{block}
%
%\begin{itemize}
%\item Past- Pre-Rap, RAP, Advisory Committee (AC) Meetings annually
%\item New DFO standard- Framework, RAP every five years, AC Annually
%\item Snow Crab Plan- Pre-Rap, Informal RAP, AC annually, Framework/Formal RAP every ~5 years
%
%\end{itemize}
%\end{block}
%\end{frame}

%----------------------------------------------------------------------------------------
%
%
%\begin{frame}
%
%\frametitle{Document Schedule}
%
% \vspace*{-0.5cm}
%\begin{block}
%
%\begin{itemize}
%\item Past- Res Doc (100+ pages) and SAR (10-15 pages) annually
%\item New DFO standard- Update document (2-3 pages) annually, Res Doc every 5 years
%\item Snow Crab Plan- Update document (10-15 pages) annually, based on SAR, Res Doc every 5 years
%\item Full assessment run annually, an items of note included in update document
%
%\end{itemize}
%
%\end{block}
%
%\end{frame}
%%------------------------------------------------
%


\end{document}

