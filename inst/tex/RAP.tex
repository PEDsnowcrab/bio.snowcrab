



%----------------------------------------------------------------------------------------
%	Presentation for snow crab RAP using beamer
% last edit: Feb 2017
%
% Note: to reduce the size of the PDF:
% use ghostscript:
% gs -q -dSAFER -dNOPAUSE -sDEVICE=pdfwrite -sPDFSETTINGS=printer -sOutputFile="resdoc2006sc.pdf" resdoc2006sc.pdf 
% the key being sPDFSETTINGS: with options: default, screen, ebook, printer, preprint
% try also:  -sCompressPages=true
% -sDownsampleColorImages=true
% -sColorImageResolution=300
% -sGrayImageResolution=300
% -sMonoImageResolution=300
% final choice to get it under 8MB :
% gs -q -dSAFER -dNOPAUSE -sDEVICE=pdfwrite -dPDFSETTINGS=/printer -dColorImageResolution=150 -dMonoImageResolution=150 -dGrayImageResolution=150  -dCompatibilityLevel=1.4 -sOutputFile="resdoc2006sc.printer.pdf" resdoc2006sc.pdf  < /dev/null
%----------------------------------------------------------------------------------------


\documentclass{beamer}

\mode<presentation> {
  %\usetheme{Hannover}
  %\usetheme{AnnArbor}
  \usetheme{Boadilla}
  %\usecolortheme{dolphin}
  \usecolortheme{seagull}
  \usefonttheme{structuresmallcapsserif}
  \beamertemplatenavigationsymbolsempty % turn off navigation
  \hypersetup{pdfstartview={Fit}} % fits the presentation to the window when first displayed
}

\usepackage{default}
%\usepackage{fourier}
\usepackage[english]{babel}															% English language/hyphenation
\usepackage[protrusion=true,expansion=true]{microtype}				% Better typography
\usepackage[toc,page]{appendix}
\usepackage[utf8]{inputenc}
\usepackage{csquotes}
\usepackage{hyperref}
\usepackage{graphicx} % Allows including images
\usepackage{graphics}
\usepackage{booktabs} % Allows the use of \toprule, \midrule and \bottomrule in tables
\usepackage{amsmath,amsfonts}				% Math packages
\numberwithin{equation}{section}		% Equationnumbering: section.eq#
\numberwithin{figure}{section}	   	% Figurenumbering: section.fig#
\numberwithin{table}{section}				% Tablenumbering: section.tab#



%----------------------------------------------------------------------------------------
% these need to be incremented each year .. <<<<<<<<<<<<<<<<<<<<<
\newcommand{\yr}{2016}
\newcommand{\yrminusone}{2015}
\newcommand{\yrminustwo}{2014}
\newcommand{\yrminusthree}{2013}
\newcommand{\yrminusfour}{2012}
\newcommand{\yrminusfive}{2011}
% these need to be incremented each year .. <<<<<<<<<<<<<<<<<<<<<


% ----------------------------------
\newcommand{\D}{.}  % to replace dots .. 
\newcommand{\bd}{\string~/bio.data}   %  \string~ is a representation of the home directory 
\newcommand{\bds}{\bd/bio.snowcrab}
\newcommand{\bdsrm}{\bds/R/maps/survey/snowcrab/annual}
\newcommand{\bdsa}{\bds/assessments}
\newcommand{\bdsay}{\bdsa/\yr}


% ----------------------------------
\title[Snow Crab Assessment \yr]{Snow Crab Assessment\\  Martimes Region\\ \yr } 
\author[Hubley, Zisserson, Cameron, Choi]{Brad Hubley, Ben Zisserson, Brent Cameron, Jae S. Choi} 
\institute[DFO Science]{
  Canadian Department of Fisheries and Oceans \\ % Your institution for the title page
  Science Branch \\
  Population Ecology Division
  \medskip
  \textit{} % Your email address
}
 

% ----------------------------------
\begin{document}


% ----------------------------------
  \begin{frame}
    \titlepage % Print the title page as the first slide
  \end{frame}
  

% ----------------------------------
  \begin{frame}
    \frametitle{Overview} % Table of contents slide
    \tableofcontents % \section{} and \subsection{} 
  \end{frame}
  
  
% ----------------------------------
  \begin{frame}
    \frametitle{4VWX Snow Crab}
    \begin{columns}[T]
      \begin{column}{0.4\textwidth}
        \begin{itemize}
          \item Southern limit of Snow Crab extent
          \item Three commercial fishing areas
        \end{itemize}
      \end{column}
      
      \begin{column}{0.6\textwidth}
        \begin{centering}
          Scotian Shelf and CFAs
          \begin{figure}
            \includegraphics[width=\textwidth]{\bds/maps/Basemap.pdf}
          \end{figure}
        \end{centering}
      \end{column}
    \end{columns}
  \end{frame}
  
  

% ----------------------------------
%
%\section{Survey}
%
%\begin{frame}
%\frametitle{Survey \yr}
%	\begin{columns}
%	\begin{column}{0.3\textwidth}
%	Survey:
%	\begin{itemize}
%	\begin{tiny}
%		\item Western component of 4x completed this year
%		\item More fishing days than normal due to poor weather conditions
%		\item Same vessel, captain and Science staff as 2014\\
%	\end{tiny}
%	\end{itemize}
%
%	Area Patterns:
%	\begin{itemize}
%	\begin{tiny}
%		\item N-ENS: Less crab pretty much everywhere in 2015, except inner Glace Bay Hole
%		\item S-ENS: Less crab inshore and on the slope edge in 2015
%		\item 4x: Slightly more in some locations, less in others. Despite suspect numbers from last year\\
%	\end{tiny}
%	\end{itemize}
%	\end{column}
%	
%	\begin{column}{0.7\textwidth}
%		\begin{figure}
%  		\includegraphics[width=\textwidth]{\bdsay/figures/Basemap_yearlydifference.pdf}
%  		\end{figure}
% 	\end{column}
% 	\end{columns}
%\end{frame}

% ----------------------------------
\section{Interactions with Other Species}
\begin{frame}
\frametitle{Interactions with Other Species}

\begin{itemize}
	\item Competitors: Competition with some benthic fish and crabs, but few strong competitors
	\item Prey: Echinoderms, shrimp, crabs, worms, bivalves, sea stars
	\item Predators: Atlantic Halibut, Atlantic Wolfish, Skates, Longhorn Cculpin, Sea Raven, Atlantic Cod, White Hake, American Plaice, and Haddock
\end{itemize}
\end{frame}

%------------------------------------------------------------------------------
\subsection{Competition/ Prey}
\begin{frame}
\frametitle{Competition/Prey}
Trends in biomass for potential predators/prey (no\textbackslash$km^2$) of Snow Crab on the Scotian Shelf, in the Snow Crab Survey
	\begin{columns}
	\begin{column}{0.25\textwidth}
	\begin{itemize}
	  \setlength\itemsep{2em}
		\item[] Jonah Crab 
		\item[] Lesser Toad Crab
		\item[] Northern Shrimp
	\end{itemize}
	\end{column}

	\begin{column}{0.25\textwidth}
 	\begin{figure}
    \includegraphics[width=\textwidth]{\bdsrm/ms.no.2511/{ms.no.2511.\yrminusone}.png}\\   
    \includegraphics[width=\textwidth]{\bdsrm/ms.no.2521/{ms.no.2521.\yrminusone}.png}\\   
    \includegraphics[width=\textwidth]{\bdsrm/ms.no.2211/{ms.no.2211.\yrminusone}.png}  
  	\end{figure}
  	\end{column}

  	\begin{column}{0.25\textwidth}
 	\begin{figure}
    \includegraphics[width=\textwidth]{\bdsrm/ms.no.2511/{ms.no.2511.\yr}.png}\\   
    \includegraphics[width=\textwidth]{\bdsrm/ms.no.2521/{ms.no.2521.\yr}.png}\\    
    \includegraphics[width=\textwidth]{\bdsrm/ms.no.2211/{ms.no.2211.\yr}.png}  
  	\end{figure}
  	\end{column}

	\begin{column}{0.25\textwidth}
 	\begin{figure}
  	\includegraphics[width=0.6\textwidth]{\bdsay/timeseries/survey/{ms.mass.2511}.pdf}\\   
    \includegraphics[width=0.6\textwidth]{\bdsay/timeseries/survey/{ms.mass.2521}.pdf}\\
    \includegraphics[width=0.6\textwidth]{\bdsay/timeseries/survey/{ms.mass.2211}.pdf}
	\end{figure}
  	\end{column}

  	\end{columns}
\end{frame}


% ----------------------------------
\subsection{Predation}
\begin{frame}
\frametitle{Potential Predation}
Trends in biomass for potential predators (no\textbackslash$km^2$) of snow crab on the Scotian Shelf, in the Snow crab survey
	\begin{columns}
	\begin{column}{0.25\textwidth}
	\begin{itemize}
	  \setlength\itemsep{2em}
		\item[] Halibut 
		\item[] Atlantic Cod 
		\item[] Thorny Skate 
	\end{itemize}
	\end{column}

	\begin{column}{0.25\textwidth}
 	\begin{figure}
    \includegraphics[width=\textwidth]{\bdsrm/ms.mass.30/{ms.mass.30.\yrminusone}.png}\\   
    \includegraphics[width=\textwidth]{\bdsrm/ms.mass.10/{ms.mass.10.\yrminusone}.png}\\   
    \includegraphics[width=\textwidth]{\bdsrm/ms.mass.201/{ms.mass.201.\yrminusone}.png}  
  	\end{figure}
  	\end{column}

  	\begin{column}{0.25\textwidth}
 	\begin{figure}
    \includegraphics[width=\textwidth]{\bdsrm/ms.mass.30/{ms.mass.30.\yr}.png}\\   
    \includegraphics[width=\textwidth]{\bdsrm/ms.mass.10/{ms.mass.10.\yr}.png}\\   
    \includegraphics[width=\textwidth]{\bdsrm/ms.mass.201/{ms.mass.201.\yr}.png}  
  	\end{figure}
  	\end{column}

	\begin{column}{0.25\textwidth}
 	\begin{figure}
  	\includegraphics[width=0.6\textwidth]{\bdsay/timeseries/survey/{ms.mass.30}.pdf}\\   
    \includegraphics[width=0.6\textwidth]{\bdsay/timeseries/survey/{ms.mass.10}.pdf}\\
    \includegraphics[width=0.6\textwidth]{\bdsay/timeseries/survey/{ms.mass.201}.pdf}
	\end{figure}
  	\end{column}

  	\end{columns}
\end{frame}




% ----------------------------------


% ----------------------------------
\subsection{Recruitment}
\begin{frame}
\frametitle{Recruitment}
\begin{columns}
	\begin{column}{0.5\textwidth}
	Size-frequency histograms of carapace width of male Snow Crab. The vertical line represents the legal size (95 mm)
	\begin{itemize}
		\item S-ENS: Stable recruitment
		\item N-ENS: A gap in recruitment, should enter the fishery in 1-2 years
		\item 4x: Minimal internal recruitment for the foreseeable future  
	\end{itemize}
	\end{column}

	\begin{column}{0.5\textwidth}
	\begin{figure}
		\includegraphics[width=\textwidth]{\bdsay/figures/size.freq/survey/male.pdf}
	\end{figure}
	\end{column}
	\end{columns}
\end{frame}


%--------------------------------------------------------------
\subsection{Reproductive Potential}
\begin{frame}
\frametitle{Sex Ratios}
	\begin{center}
	Proportion of females in the mature population
	\end{center}
	\begin{columns}
	\begin{column}{0.3\textwidth}
	\begin{itemize}
	\begin{footnotesize}
		\item S-ENS: Stable, much higher inshore this year...highest since 2008
		\item N-ENS: Recovering from near zero
		\item 4x: Continued high proportion of males  
	\end{footnotesize}
	\end{itemize}
	\end{column}

	\begin{column}{0.35\textwidth}
	\begin{figure}
	\includegraphics[width=\textwidth]{\bds/R/maps/survey/snowcrab/annual/{sexratio.mat}/{sexratio.mat.\yr}.png}
	\end{figure}
	\end{column}

	\begin{column}{0.35\textwidth}
	\begin{figure}
	 \includegraphics[width=\textwidth]{\bdsay/timeseries/survey/{sexratio.mat}.pdf}
	\end{figure}
	\end{column}
	\end{columns}
\end{frame}


% ----------------------------------
\begin{frame}
\frametitle{Female Size-Frequency}
\begin{columns}
	\begin{column}{0.5\textwidth}
		\begin{itemize}
		\item Size-frequency histograms of carapace width of female Snow Crab
		\item Newly matured female crab are expected in all areas for the next 3-4 years
		\item Each newly matured female should support egg production for 3-5 years
		\end{itemize}
	\end{column}

	\begin{column}{0.5\textwidth}
	\begin{figure}
		\includegraphics[width=\textwidth]{\bdsay/figures/{size.freq}/survey/female.pdf}
	\end{figure}
	\end{column}
	\end{columns}
\end{frame}


% ----------------------------------
\begin{frame}
\frametitle{Primiparous/Multiparous}
\begin{columns}
	\begin{column}{0.5\textwidth}
	\begin{center}
	Primiparous 
	\end{center}
		\begin{figure}
		\includegraphics[width=\textwidth]{\bds/R/maps/survey/snowcrab/annual/{totno.female.primiparous}/{totno.female.primiparous.\yr}.png}
	\end{figure}
	\end{column}

	\begin{column}{0.5\textwidth}
	\begin{center}
	Multiparous 
	\end{center}
	\begin{figure}
		\includegraphics[width=\textwidth]{\bds/R/maps/survey/snowcrab/annual/{totno.female.multiparous}/{totno.female.multiparous.\yr}.png}	
	\end{figure}
	\end{column}
	\end{columns}
\end{frame}


% ----------------------------------
\begin{frame}
\frametitle{Mature Females}
	\begin{center}
	Total number of females in the mature population. Reproductive potential peaked in 2007/8 and continues to remain low (except 4x in 2012).
	\end{center}
	\begin{columns}
    \begin{column}{0.5\textwidth}
      \begin{figure}
        \includegraphics[width=\textwidth]{\bds/R/maps/survey/snowcrab/annual/totno.female.mat/{totno.female.mat.\yr}.png}	
      \end{figure}
    \end{column}

  	\begin{column}{0.5\textwidth}
      \begin{figure}
        \includegraphics[width=0.7\textwidth]{\bdsay/timeseries/survey/{totno.female.mat}.pdf}
      \end{figure}
    \end{column}
  \end{columns}
\end{frame}


% ----------------------------------
\subsection{Fishable Biomasss Geometric Mean}
\begin{frame}
\frametitle{Fishable Biomass Index -- Geometric Mean}
\begin{columns}
\begin{column}{0.5\textwidth}
	\begin{center}
	Fishable Biomass from Annual Snow Crab Survey. (t\textbackslash$km^2$).\\ 
	\end{center}
	\begin{figure}
		\includegraphics[width=0.9\textwidth]{\bds/R/maps/survey/snowcrab/annual/R0.mass/{R0.mass.\yr}.png}
	\end{figure}
\end{column}
\begin{column}{0.5\textwidth}
	\begin{center}
	Time series of the Fishable Biomass from Annual Snow Crab Survey (Geometric Mean)
	\end{center}
\begin{figure}
    \centering
    \includegraphics[width=0.7\textwidth]{\bdsay/timeseries/survey/{R0.mass}.pdf}
 \end{figure}
\end{column}
\end{columns}
\end{frame}


% ----------------------------------
\section{Lattice Based Models (LBM)}
\begin{frame}
\frametitle{Space Time Modelling -- Lattice Based Models}

\begin{itemize}
	\item Latest interation of space time modelling for determining snowcrab abundance using a 3 stage process
	\item 1st Stage: Global influence of environmental and ecological factors using a GAM
	\item 2nd Stage: Local timeseries analysis with seasonal and annual harmonic components for each location
	\item 3rd Stage: Kriging to account for local spatial variability at each location and time
	\item Results combine to form prediction surfaces for snowcrab habitat and abundance (non-zero data) 
	\item Habitat is then used as weights for non-zero biomass to get final estimate of fishable biomass 
\end{itemize}
\end{frame}

%------------------------------------------------------------------------------
\subsection{Ecosystem inputs}
\begin{frame}
\frametitle{Ecosystem inputs}
Environmental factors: Temperature, Depth and Substrate
	\begin{columns}
	\begin{column}{0.25\textwidth}
	\begin{itemize}
	  \setlength\itemsep{2em}
		\item[] Bottom Temperature 
		\item[] Bottom Temperature SD
		\item[] Substrate ln(grain size)
	\end{itemize}
	\end{column}

	\begin{column}{0.25\textwidth}
 	\begin{figure}
    \includegraphics[width=\textwidth]{\bd/bio.temperature/maps/SSE/bottom.predictions/climatology/{temperatures.bottom}.png}\\   
    \includegraphics[width=\textwidth]{\bd/bio.temperature/maps/SSE/bottom.predictions/climatology/{temperatures.bottom.sd}.png}\\   
    \includegraphics[width=\textwidth]{\bd/bio.substrate/R/{substrate.grainsize.edit}.png}  
  	\end{figure}
  	\end{column}

  	\begin{column}{0.25\textwidth}
 	\begin{figure}
    \includegraphics[width=\textwidth]{\bd/bio.bathymetry/maps/SSE/depth.png}\\   
    \includegraphics[width=\textwidth]{\bd/bio.bathymetry/maps/SSE/slope.png}\\   
    \includegraphics[width=\textwidth]{\bd/bio.bathymetry/maps/SSE/curvature.png}  
  	\end{figure}
  	\end{column}

	\begin{column}{0.25\textwidth}
	\begin{itemize}
	  \setlength\itemsep{2em}
		\item[] Depth
		\item[] Slope
		\item[] Curvature
	\end{itemize}
  	\end{column}

  	\end{columns}
\end{frame}

\subsection{Ecosystem inputs}
\begin{frame}
\frametitle{Ecosystem inputs}
Environmental factors: Annual Estimates of Bottom Temperature
	\begin{columns}
	\begin{column}{0.25\textwidth}
	\begin{itemize}
	  \setlength\itemsep{4em}
		\item[]  \yrminusfive
		\item[]  \yrminusthree
		\item[]  \yrminusone
	\end{itemize}
	\end{column}

	\begin{column}{0.25\textwidth}
 	\begin{figure}
    \includegraphics[width=\textwidth]{\bd/bio.temperature/maps/SSE/bottom.predictions/annual/{temperatures.bottom.\yrminusfive}.png}\\   
    \includegraphics[width=\textwidth]{\bd/bio.temperature/maps/SSE/bottom.predictions/annual/{temperatures.bottom.\yrminusthree}.png}\\   
    \includegraphics[width=\textwidth]{\bd/bio.temperature/maps/SSE/bottom.predictions/annual/{temperatures.bottom.\yrminusone}.png}\\   
  	\end{figure}
  	\end{column}

  	\begin{column}{0.25\textwidth}
 	\begin{figure}
    \includegraphics[width=\textwidth]{\bd/bio.temperature/maps/SSE/bottom.predictions/annual/{temperatures.bottom.\yrminusfour}.png}\\   
    \includegraphics[width=\textwidth]{\bd/bio.temperature/maps/SSE/bottom.predictions/annual/{temperatures.bottom.\yrminustwo}.png}\\   
    \includegraphics[width=\textwidth]{\bd/bio.temperature/maps/SSE/bottom.predictions/annual/{temperatures.bottom.\yr}.png}\\   
  	\end{figure}
  	\end{column}

	\begin{column}{0.25\textwidth}
	\begin{itemize}
	  \setlength\itemsep{4em}
		\item[]  \yrminusfour
		\item[]  \yrminustwo
		\item[]  \yr
	\end{itemize}
  	\end{column}

  	\end{columns}
\end{frame}

\begin{frame}
\frametitle{Ecosystem inputs}
Ecological factors: Species Composition, Species Diversity and Metabolic Rate
	\begin{columns}
	\begin{column}{0.25\textwidth}
	\begin{itemize}
	  \setlength\itemsep{2em}
		\item[] Species Composition CA1 
		\item[] Predicted number of species
		\item[] Total Metabolic rate
	\end{itemize}
	\end{column}

	\begin{column}{0.25\textwidth}
 	\begin{figure}
    \includegraphics[width=\textwidth]{\bd/bio.indicators/speciescomposition/maps/ca1/snowcrab/annual/{ca1.mean.\yr}.png}\\   
    \includegraphics[width=\textwidth]{\bd/bio.indicators/speciesarea/maps/Npred/snowcrab/annual/{Npred.mean.\yr}.png}\\   
    \includegraphics[width=\textwidth]{\bd/bio.indicators/metabolism/maps/mr/snowcrab/annual/{mr.mean.\yr}.png}  
  	\end{figure}
  	\end{column}

  	\begin{column}{0.25\textwidth}
 	\begin{figure}
    \includegraphics[width=\textwidth]{\bd/bio.indicators/speciescomposition/maps/ca2/snowcrab/annual/{ca2.mean.\yr}.png}\\   
    \includegraphics[width=\textwidth]{\bd/bio.indicators/speciesarea/maps/Z/snowcrab/annual/{Z.mean.\yr}.png}\\    
    \includegraphics[width=\textwidth]{\bd/bio.indicators/metabolism/maps/smr/snowcrab/annual/{smr.mean.\yr}.png}  
  	\end{figure}
  	\end{column}

	\begin{column}{0.25\textwidth}
	\begin{itemize}
	  \setlength\itemsep{2em}
		\item[] Species Composition CA2 
		\item[] slope of species area relationship (diversity)
		\item[] Specific Metabolic rate
	\end{itemize}
  	\end{column}

  	\end{columns}
\end{frame}


\subsection{Fishable Biomass LBM}
\begin{frame}
\frametitle{Fishable Biomass Index -- LBM}
\begin{columns}

\begin{column}{0.5\textwidth}
  \begin{center}
    Viable habitat probability for \yr, probility of snowcrab presence
  \end{center}
  \begin{figure}
    \includegraphics[width=0.9\textwidth]{\bds/maps/snowcrab.large.males_presence_absence/snowcrab/annual/{snowcrab.large.males_presence_absence.mean.\yr}.png}
    \end{figure}
\end{column}

\begin{column}{0.5\textwidth}
  \begin{center}
    Interpolated fishable biomass with non-zero data ln(t per km sq) 
  \end{center}
  \begin{figure}
    \includegraphics[width=0.9\textwidth]{\bds/maps/snowcrab.large.males_abundance/snowcrab/annual/{snowcrab.large.males_abundance.mean.\yr}.png}
  \end{figure}
\end{column}


\end{columns}
\end{frame}


%------------------------------------------------
\begin{frame}
  \frametitle{Fishable Biomass Index -- LBM}
   \begin{columns} 
    \begin{column}{0.5\textwidth}
      \begin{center}
        Estimated fishable abundance\\ 
        ln(t per km sq) 
      \end{center}
      \begin{figure}
        \includegraphics[width=0.9\textwidth]{\bds/maps/fishable.biomass/snowcrab/{prediction.abundance.mean.\yr}.png}
      \end{figure}
    \end{column}
    
    \begin{column}{0.5\textwidth}
      \begin{center}
        Time series of the Area Expanded Fishable Biomass Density
      \end{center}
      \begin{figure}
        \centering
        \includegraphics[width=0.7\textwidth]{\bdsay/timeseries/interpolated/{snowcrab.biomass.index}.png}
      \end{figure}
    \end{column}
    \end{columns}
\end{frame}


%------------------------------------------------
\section{Fisheries Model}
\begin{frame}
  \tableofcontents[currentsection]
\end{frame}


% ----------------------------------
\subsection{Fishable Biomass -- Biomass dynamics}
\begin{frame}
\frametitle{Fishable Biomass -- Biomass dynamics}
\begin{columns} 
\begin{column}{0.5\textwidth}
 \begin{itemize}
  \begin{tiny}
     \item Red: The fishable biomass index
     \item Blue: The posterior mean fishable biomass estimated from the logistic model
     \item Grey: The density distribution of posterior fishable biomass estimates darkest area being medians and the 95\% Credible Intervals (CI). 
     \end{tiny}
    \end{itemize}
\end{column}
\begin{column}{0.5\textwidth}
  \begin{figure}[ht]
    \centering
    \includegraphics[width=0.9\textwidth]{\bdsay/{biomass.timeseries}.png}
  \end{figure}
\end{column}
\end{columns} 

\end{frame}


%------------------------------------------------
\subsection{Fishing Mortality}
\begin{frame}
\frametitle{Fishing Mortality}
\begin{columns}
\begin{column}{0.4\textwidth}
Mortality factors are associated with:
\begin{itemize}
\begin{footnotesize}
	\item Predation: Long-term risk if increase in predators
	\item Food limitation: Likely stable 
	\item Competition: Likely stable\\
\end{footnotesize}
\end{itemize}

Area trends:
\begin{itemize}
\begin{footnotesize}
	\item N-ENS: Increasing over the past several years
	\item S-ENS: Slightly above 20\% harvest rate
	\item 4x: Peaking in 2005 and 2011/2012
\end{footnotesize}
\end{itemize}
\end{column}
\begin{column}{0.6\textwidth}
\begin{footnotesize}
Fishing mortality
\end{footnotesize}

\begin{itemize}
\begin{tiny}
	\item[] Grey Area: Posterior density distributions, with median with 90\% CI 
	\item[] Dark-dashed line: is the 20\% harvest rate
	\item[] Red line: estimated FMSY 	
\end{tiny}
\end{itemize}

\begin{figure}
    \includegraphics[width=0.8\textwidth]{\bdsay/{fishingmortality.timeseries}.png}
\end{figure}
\end{column}

\end{columns}
\end{frame}


% ----------------------------------
\subsection{Posteriors}
\begin{frame}
  \frametitle{Posteriors}
  \begin{figure}
    \centering
    \includegraphics[width=0.6\textwidth]{\bdsay/{K.density}.png} 
  \end{figure}
\end{frame}


% ----------------------------------
\begin{frame}
  \frametitle{Posteriors}
  \begin{figure}
    \centering
    \includegraphics[width=0.6\textwidth]{\bdsay/{r.density}.png} 
  \end{figure}
\end{frame}


% ----------------------------------
\begin{frame}
  \frametitle{Posteriors}
  \begin{figure}
    \centering
    \includegraphics[width=0.6\textwidth]{\bdsay/{q.density}.png} 
  \end{figure}
\end{frame}


% ----------------------------------
\begin{frame}
  \frametitle{Posteriors}
  \begin{figure}
    \centering
    \includegraphics[width=0.6\textwidth]{\bdsay/{bp.sd.density}.png} 
  \end{figure}
\end{frame}


% ----------------------------------
\begin{frame}
  \frametitle{Posteriors}
  \begin{figure}
    \centering
    \includegraphics[width=0.6\textwidth]{\bdsay/{bo.sd.density}.png} 
  \end{figure}
\end{frame}

\subsection{Harvest Advice}

% ----------------------------------
\begin{frame}
  \frametitle{Precautionary approach}
      \begin{figure}[ht]
     \centering
     \includegraphics[width=0.8\textwidth]{\bdsa/common/hcr.png}
     \end{figure}
\end{frame}


% ----------------------------------
\begin{frame}
  \frametitle{Precautionary approach}
\begin{figure}
  \centering
  \includegraphics[width=0.6\textwidth]{\bdsay/{hcr.default}.png}\\ 
\end{figure}
\end{frame}

\begin{frame}
\frametitle{Harvest Advice}

\begin{itemize}
	\item N-ENS
	\begin{itemize}
		\item Stock fell rapidly from 2013 to 2014 into the "cautious zone" prompting significant deceases in TAC
		\item New more robust methodology indicates the stock has responded to management action and is now in the "heathly zone"
		\item Recommend an increase in TAC
	\end{itemize}
	\item S-ENS
	\begin{itemize}
		\item New more robust methodology indicates declining biomass and increasing F since 2013
		\item Stock is in the "heathly zone" but close to "cautious zone"
		\item Recommend a decrease in TAC
	\end{itemize}
	\item 4X
	\begin{itemize}
		\item Erratic temperature fields constrict habitat in this marginal area
		\item Stock is in the "cautious zone"
		\item Recommend status quo or a modest increase in TAC
	\end{itemize}
\end{itemize}
\end{frame}

% ----------------------------------
% ----------------------------------
% ----------------------------------

\end{document}

