\documentclass[11pt]{article}
\usepackage{graphicx}
\usepackage{subfig}
%\usepackage{pdfcomment}
\usepackage{amsmath}


\usepackage[top=2.4cm, bottom=2.4cm, left=3cm, right=3cm]{geometry}
\usepackage{fancyhdr}
\pagestyle{fancy}

\lhead{\bf Maritimes Region}
\rhead{\bf Scotian Shelf Snow Crab 2017}
%\lfoot{edited on Feb 21, 2018}
\cfoot{\thepage}
\renewcommand{\headrulewidth}{0.4pt}
\renewcommand{\footrulewidth}{0.4pt}
\newcommand{\D}{.}
\newcommand{\h}{C:/} %home directory
\newcommand{\e}{bio.data/}
\newcommand{\es}{bio.data/bio.snowcrab/} %snow crab assessment base directory
\newcommand{\eb}{bathymetry/}
\newcommand{\et}{bio.data/bio.temperature/}
\newcommand{\ea}{bio.data/aegis/}
\newcommand{\ei}{bio.data/bio.indicators/}
\newcommand{\Ay}{assessments/2017/}
\newcommand{\A}{assessments/}
\newcommand{\m}{output/maps/} %base folder for maps
\newcommand{\mb}{output/maps/survey/snowcrab/annual/bycatch/} %survey bycatch folder
\newcommand{\ebugs}{bio.data/bio.snowcrab/assessments/2016/figures/bugs/survey/}

\setlength{\headheight}{15pt}

% these need to be incremented each year .. <<<<<<<<<<<<<<<<<<<<<
\newcommand{\yr}{2017}
\newcommand{\yrminusone}{2016}
\newcommand{\yrminustwo}{2015}
\newcommand{\yrminusthree}{2014}
\newcommand{\yrminusfour}{2013}
\newcommand{\yrminusfive}{2012}
\newcommand{\yrminussix}{2011}
\newcommand{\yrminusseven}{2010}



\begin{document}

\begin{figure}
    \includegraphics[width=1.1\textwidth]{\h \es \A common/basemap.png}
    \caption{Location of geographic areas and the management areas for snow crab on the Scotian Shelf.}
  \end{figure}
\clearpage



%--------------fishing effort------------------------------------

\begin{figure}
    \includegraphics[width=\textwidth]{\h \es \Ay timeseries/fishery/effort\D ts.pdf}
    \caption{Temporal variations in the fishing effort for snow crab on the Scotian Shelf, expressed as the number of trap hauls. Year in 4X refers to the year at the start of the fishing season.}
\end{figure}
\clearpage


%--------------landings------------------------------------

\begin{figure}
    \includegraphics[width=\textwidth]{\h \es \Ay timeseries/fishery/landings\D ts.pdf}
    \caption{Temporal variations in the landings of snow crab on the Scotian Shelf (t). Note the sharp increase in landings associated with dramatic increases to total allowable catches (TACs) and a doubling of fishing effort in the year 2000. The landings follow the TACs with little deviation (table 2-4). Year in 4X refers to the year at the start of the fishing season.}
\end{figure}
\clearpage


%--------------catch rates------------------------------------

\begin{figure}
    \includegraphics[width=\textwidth]{\h \es \Ay timeseries/fishery/cpue\D ts.pdf}
    \caption{Temporal variations in catch rates of snow crab on the Scotian Shelf, expressed as kg per trap haul. Trap design and size have changed over time. No correction for these varying trap-types nor soak time and bait-type has been attempted (see Methods). Year in 4X refers to the year at the start of the fishing season.}
\end{figure}
\clearpage


%--------------at sea observer coverage------------------------------------

\begin{figure}
    \centering
   \subfloat{\includegraphics[width=0.6\textwidth]{\h \es \m observer\D locations/observer\D locations\D \yrminustwo.png}}\\
  \subfloat{\includegraphics[width=0.6\textwidth]{\h \es \m observer\D locations/observer\D locations\D \yrminusone.png}}\\
    \subfloat{\includegraphics[width=0.6\textwidth]{\h \es \m observer\D locations/observer\D locations\D \yr.png}}\\
  \caption{Snow crab fishing locations monitored by at-sea-observers on the Scotian Shelf during each of the past three fishing seasons.}
\end{figure}
\clearpage


%--------------snow crab survey locations------------------------------------

\begin{figure}
\centering
  \subfloat{\includegraphics[width=0.6\textwidth]{\h \es \m survey\D locations/survey\D locations\D \yrminustwo.png}}\\
  \subfloat{\includegraphics[width=0.6\textwidth]{\h \es \m survey\D locations/survey\D locations\D \yrminusone.png}}\\
  \subfloat{\includegraphics[width=0.6\textwidth]{\h \es \m survey\D locations/survey\D locations\D \yr.png}}\\
	\caption{Locations of snow crab survey trawl sets on the Scotian Shelf during each of the past three years. Note stations not completed in Southeast corner for 2017 survey as compared to other years.}
\end{figure}
\clearpage


%--------------snow crab growth curves------------------------------------

\begin{figure}
    \centering
   \subfloat{\includegraphics[width=0.7\textwidth]{\h \es \A common/male\D growth\D cw.pdf}}\\
\subfloat{\includegraphics[width=0.7\textwidth]{\h \es \A common/male\D growth\D mass.pdf}}\\
\caption{Growth curves determined from modal length frequency analysis of male snow crab on the Scotian Shelf.}
\end{figure}
\clearpage


%--------------scotian shelf bottom characteristics------------------------------------


\begin{figure}
\centering
%\subfloat{\includegraphics[width=0.45\textwidth,trim={0 1cm 0 1cm}]{\h \ea temperature/maps/SSE/bottom\D predictions/climatology/temperatures\D bottom.png}}\ 
\subfloat{\includegraphics[width=0.65\textwidth,trim={0 1cm 0 1cm}]{\h \ea \eb maps/SSE/depth.png}}\\
%\subfloat{\includegraphics[width=0.45\textwidth,trim={0 1cm 0 1cm}]{\h \ea temperature/maps/SSE/bottom\D predictions/climatologys\D bottom\D sd.png}}\
\subfloat{\includegraphics[width=0.65\textwidth,trim={0 1cm 0 1cm}]{\h \ea \eb maps/SSE/slope.png}}\\
%\subfloat{\includegraphics[width=0.45\textwidth,trim={0 1cm 0 1cm}]{\h \ea temperature/maps/SSE/bottom\D predictions/climatology\D bottom\D amplitude.png}}\ 
\subfloat{\includegraphics[width=0.65\textwidth,trim={0 1cm 0 1cm}]{\h \ea \eb maps/SSE/curvature.png}}\\ 
\caption{Habitat characteristics used for modeling snow crab habitat delineation. The visualizations of temperature variations are for climatological means. Annual temperature variation estimates were used for modeling.}
\end{figure}
\clearpage

%--------------scotian shelf community characteristics------------------------------------
% Need to adjust caption to better explain which plots are shown
\begin{figure}
  \centering  
  	\subfloat{\includegraphics[width=0.9\textwidth,trim={0 1cm 0 1cm}]{\h \ea speciescomposition/maps/pca1/snowcrab/annual/pca1\D mean\D 2016.png}}\\
	\subfloat{\includegraphics[width=0.9\textwidth,trim={0 1cm 0 1cm}]{\h \ea speciescomposition/maps/pca2/snowcrab/annual/pca2\D mean\D 2016.png}}\
%	\subfloat{\includegraphics[width=0.45\textwidth,trim={0 1cm 0 1cm}]{\h \ea speciesarea/maps/Npred/snowcrab/annual/Npred\D mean\D 2016.png}}\	  
%	\subfloat{\includegraphics[width=0.45\textwidth,trim={0 1cm 0 1cm}]{\h \ea speciesarea/maps/Z/snowcrab/annual/Z\D mean\D 2016.png}}\\  
%	\subfloat{\includegraphics[width=0.45\textwidth,trim={0 1cm 0 1cm}]{\h \ea metabolism/maps/mr/snowcrab/annual/mr\D mean\D 2016.png}}\
%	\subfloat{\includegraphics[width=0.45\textwidth,trim={0 1cm 0 1cm}]{\h \ea metabolism/maps/smr/snowcrab/annual/smr\D mean\D 2016.png}}\\
	\caption{Principal component analysis of species composition (community) characteristics on the Scotian Shelf used in snow crab habitat determination modelling. Annual time series are used. Top figure is the first axis or ordination, bottom figure is second axis of ordination.}
\end{figure}
\clearpage


%--------------scotian shelf community characteristics------------------------------------
% Not used in 201- BZ
%\begin{figure}
%\centering
%\includegraphics[width=0.9\textwidth]{\h \es assessments/2013/gamhabitat.png}
%\caption{GAM model terms for the community characteristics used to describe the habitat area for the %fishable component of the Scotian Shelf snow crab population.}
%\end{figure}
%\clearpage

%--------------predictive habitat maps------------------------------------

\begin{figure}
\centering
\subfloat{\includegraphics[width=0.49\textwidth]{\h \es maps/snowcrab\D large\D males_presence_absence/snowcrab/annual/snowcrab\D large\D males_presence_absence\D mean\D \yrminusfive.png}}\
\subfloat{\includegraphics[width=0.49\textwidth]{\h \es maps/snowcrab\D large\D males_presence_absence/snowcrab/annual/snowcrab\D large\D males_presence_absence\D mean\D \yrminusfour.png}}\\
\subfloat{\includegraphics[width=0.49\textwidth]{\h \es maps/snowcrab\D large\D males_presence_absence/snowcrab/annual/snowcrab\D large\D males_presence_absence\D mean\D \yrminusthree.png}}\
\subfloat{\includegraphics[width=0.49\textwidth]{\h \es maps/snowcrab\D large\D males_presence_absence/snowcrab/annual/snowcrab\D large\D males_presence_absence\D mean\D \yrminustwo.png}}\\
\subfloat{\includegraphics[width=0.49\textwidth]{\h \es maps/snowcrab\D large\D males_presence_absence/snowcrab/annual/snowcrab\D large\D males_presence_absence\D mean\D \yrminusone.png}}\
\subfloat{\includegraphics[width=0.49\textwidth]{\h \es maps/snowcrab\D large\D males_presence_absence/snowcrab/annual/snowcrab\D large\D males_presence_absence\D mean\D \yr.png}}\\
\caption{Annual interpolations of potential habitat for the fishable component of SSE snow crab represented as the probability of finding snow crab. Spatial representations are generated with stmv modelling using generalized additive models of several habitat, environmental and biological variables.}
\end{figure}
\clearpage


%--------------prediction of commercial male biomass from gam------------------------------------
% Need to confirm these are proper figures before publication

\begin{figure}
\centering
\subfloat{\includegraphics[width=0.49\textwidth]{\h \es maps/fishable\D biomass/snowcrab/prediction\D abundance\D mean\D \yrminusfive.png}}\
\subfloat{\includegraphics[width=0.49\textwidth]{\h \es maps/fishable\D biomass/snowcrab/prediction\D abundance\D mean\D \yrminusfour.png}}\\
\subfloat{\includegraphics[width=0.49\textwidth]{\h \es maps/fishable\D biomass/snowcrab/prediction\D abundance\D mean\D \yrminusthree.png}}\
\subfloat{\includegraphics[width=0.49\textwidth]{\h \es maps/fishable\D biomass/snowcrab/prediction\D abundance\D mean\D \yrminustwo.png}}\\
\subfloat{\includegraphics[width=0.49\textwidth]{\h \es maps/fishable\D biomass/snowcrab/prediction\D abundance\D mean\D \yrminusone.png}}\
\subfloat{\includegraphics[width=0.49\textwidth]{\h \es maps/fishable\D biomass/snowcrab/prediction\D abundance\D mean\D \yr.png}}\\
\caption{Annual interpolations of fishable snow crab biomass $log\left(\tfrac{t}{km^2}\right)$. Spatial representations are generated with stmv modelling using generalized additive models of several habitat, environmental and biological variables.}
\end{figure}
\clearpage


%------groundfish survey strata for diet analysis
% Not used in 2018- BZ

%\begin{figure}
%\centering
%\includegraphics[width=\textwidth]{\h \es assessments/common/groundfish\D crab\D areas.png}
%\caption{The DFO summer groundfish survey strata overlaid with the snow crab fishing areas. Trends in groundfish biomass indices for N-ENS were captured using the data from strata 440-442, S-ENS 443-467 and 4X 470-483.}
%\end{figure}
%\clearpage


%--------------diagram of snowcrab growth stanzas------------------------------------
\begin{figure}
\centering
\includegraphics[width=\textwidth]{\h \es \A common/sncrab\D growth\D stanzas.png}
\caption{The growth stanzas of male snow crab. Each instar is determined from CW bounds obtained from modal analysis and categorized to carapace condition (CC) and maturity from visual inspection and/or maturity equations. Snow crab are resident in each growth stanza for 1 year, with the exception of CC2 to CC4 which are known from mark-recapture studies to last from three to five years.}
\end{figure}
\clearpage


%--------------sorted ordination------------------------------------
%--------------Removed 2018 as completion of the table becoming near imposible due to loss of data streams, etc. BZ
%\begin{figure}
%\centering
%\includegraphics[width=\textwidth]{\h \ei output/keyfactors\D anomalies.pdf}
%\caption{Sorted ordination of anomalies of key social, economic and ecological patterns on the Scotian Shelf relevant to snow crab. Red indicates below the mean and green indicates above the mean.}
%\end{figure}
%\clearpage


%--------------first axis of sorted ordination------------------------------------
%-------------- Removed 2018. BZ

%\begin{figure}
%\centering
%\includegraphics[width=\textwidth]{\h \ei output/keyfactors\D PC1.pdf}
%\caption{First axis of variations in ordination of anomalies of social, economic and ecological patterns on the Scotian Shelf. Note strong variability observed near the time of the fishery collapse in the early 1990s. Note strong variability observed near the time of the fishery collapse in the early 1990s.}
%\end{figure}
%\clearpage


%------------------Movement map of spaghetti tags

\begin{figure}
\centering
\subfloat{\includegraphics[width=0.6\textwidth]{\h \es \Ay figures/other\D figs/tagging/spaghetti.png}}\
\caption{Movement of tagged terminally moulted snow crab on the Scotian Shelf. Movement path between release and recapture locations constrained to the shortest path within depth contours of 60 and 280 m. Circles represent release locations and colours represent time interval (in years) between initial tagging and last recapture.}
\end{figure}
\clearpage


%------------------------ Displacement vs. time for spaghetti tags

\begin{figure}
\centering
\subfloat{\includegraphics[width=0.9\textwidth]{\h \es \Ay figures/other\D figs/tagging/distdays.pdf}}\
\caption{Distance travelled vs. time to capture for tagged snow crab on the Scotian Shelf since 2004. Data grouped by release year with release area distinguished by color. Two separate movement patterns observed, most apparent in NENS for the 2009-2013 release group. Periodicity in time intervals are explained by recaptures occuring during seasonal fishing operations.}
\end{figure}


%------------------------Movement and return rates of spaghetti tags

\begin{figure}
    \centering
    \subfloat{\includegraphics[width=0.6\textwidth]{\h \es \Ay figures/other\D figs/tagging/speed.pdf}}\\
    \subfloat{\includegraphics[width=0.6\textwidth]{\h \es \Ay figures/other\D figs/tagging/returns.pdf}}\\
\caption{(Top) Mean rate of movement of snow crab tagged on the Scotian Shelf by area and year. Route lengths derived from calculated shortest paths constrained by depth range of 60-280 m. Small sample size and short time between mark and potential recapture account for the higher than normal rates for S-ENS in 2016 and 2017. (Bottom) Tag return rate, number of returns from tags applied in given area and year.}
\end{figure}
\clearpage


%------------------------Acoustic tag movement map

\begin{figure}
\centering
\subfloat{\includegraphics[width=0.95\textwidth]{\h \es \Ay figures/other\D figs/tagging/acoustic.png}}\
\caption{Movement of acoustic tagged snow crab on the Scotian Shelf. Movement path between mark and detection locations constrained to the shortest path within depth contours of 60 and 280 m. Triangles represent release locations and individual colours represent individual tagged animals.}
\end{figure}
\clearpage

%------------------------time series of observed temperatures snowcrab and groundfish surveys

\begin{figure}
\centering
\includegraphics[width=1.0\textwidth]{\h \es \Ay timeseries/survey/t.pdf} %
\caption{Annual variations in bottom temperature observed during the ENS snow crab survey. The horizontal line indicates the long-term median temperature within each subarea. Error bars are 1 standard deviation. }
\end{figure}
\clearpage


\begin{figure}
\centering
\includegraphics[width=1.0\textwidth]{\h \es \Ay timeseries/survey/groundfish\D t.pdf} %
\caption{Annual variations in bottom temperature observed during the DFO July RV Groundfish Survey. The horizontal line indicates the long-term median temperature within each subarea. Error bars are 1 standard deviation.}
\end{figure}
\clearpage


%%--------------bottom tempaerture maps------------------------------------


\begin{figure}
\centering

\subfloat{\includegraphics[width=0.49\textwidth,trim={0 1cm 0 1cm}]{\h \ea temperature/maps/SSE/bottom\D predictions/annual/temperatures\D bottom\D  1960.png}}\ 
\subfloat{\includegraphics[width=0.49\textwidth,trim={0 1cm 0 1cm}]{\h \ea temperature/maps/SSE/bottom\D predictions/annual/temperatures\D bottom\D  1980.png}}\\ 
\subfloat{\includegraphics[width=0.49\textwidth,trim={0 1cm 0 1cm}]{\h \ea temperature/maps/SSE/bottom\D predictions/annual/temperatures\D bottom\D  2000.png}}\ 
\subfloat{\includegraphics[width=0.49\textwidth,trim={0 1cm 0 1cm}]{\h \ea temperature/maps/SSE/bottom\D predictions/annual/temperatures\D bottom\D  \yrminusfour.png}}\\ 
\subfloat{\includegraphics[width=0.49\textwidth,trim={0 1cm 0 1cm}]{\h \ea temperature/maps/SSE/bottom\D predictions/annual/temperatures\D bottom\D  \yrminusthree}}\ 
\subfloat{\includegraphics[width=0.49\textwidth,trim={0 1cm 0 1cm}]{\h \ea temperature/maps/SSE/bottom\D predictions/annual/temperatures\D bottom\D  \yrminustwo.png}}\\ 
\subfloat{\includegraphics[width=0.49\textwidth,trim={0 1cm 0 1cm}]{\h \ea temperature/maps/SSE/bottom\D predictions/annual/temperatures\D bottom\D  \yrminusone.png}}\ 
\subfloat{\includegraphics[width=0.49\textwidth,trim={0 1cm 0 1cm}]{\h \ea temperature/maps/SSE/bottom\D predictions/annual/temperatures\D bottom\D  \yr.png}}\\ 
\caption{Interpolated mean annual bottom temperatures on the Scotian Shelf for selected years. These interpolations use all available water temperature data collected in the area including Groundfish Surveys, Snow crab survey, and AZMP monitoring stations.}
\end{figure}
\clearpage


%%-----------------------------potential snow crab habitat------------------------

\begin{figure}
\centering
\includegraphics[width=\textwidth]{\h \es \Ay timeseries/interpolated/snowcrab\D habitat\D sa.png}
\caption{Annual variations in the surface area of potential snow crab habitat. The horizontal line indicates the long-term median surface area within each subarea. }
\end{figure}
\clearpage


%-----------------------------temperature in potential snow crab habitat------------------------
% Not used in 2018. BZ

%\begin{figure}
%\centering
%\includegraphics[width=0.9\textwidth]{\h \es assessments/2013/timeseries/interpolated/mean\D bottom\D temp\D snowcrab\D habitat.png} %
%\caption{Annual variations in bottom temperature within potential snow crab habitat. The horizontal line indicates the long-term arithmetic mean temperature within each subarea. Error %bars are 1 standard deviation. See caveats previous Figure. Note increasing in all areas since the mid-2001s, especially in area 4X. The current mean temperature in 4X is %above the temperature requirements for snow crab.}
%\end{figure}
%\clearpage
 
%--------------------------halibut as predators------------------------

\begin{figure}
\centering
\subfloat{\includegraphics[width=0.49\textwidth]{\h \es \mb ms\D no\D 30/ms\D no\D 30\D \yrminusthree.png}}\ 
\subfloat{\includegraphics[width=0.49\textwidth]{\h \es \mb ms\D no\D 30/ms\D no\D 30\D \yrminustwo.png}}\\ 
\subfloat{\includegraphics[width=0.49\textwidth]{\h \es \mb ms\D no\D 30/ms\D no\D 30\D \yrminusone.png}}\ 
\subfloat{\includegraphics[width=0.49\textwidth]{\h \es \mb ms\D no\D 30/ms\D no\D 30\D \yr.png}}\\ 
\caption{Locations of potential predators of snow crab on the Scotian Shelf: \textbf{Atlantic halibut}. Scale is $\tfrac{number}{km^2}$.}
\end{figure}
\clearpage

%------------------------time series of halibut abundance

\begin{figure}
\centering
\includegraphics[width=0.9\textwidth]{\h \es \Ay timeseries/survey/ms\D mass\D 30.pdf}
\caption{Trends in biomass $\left(\tfrac{t}{km^2}\right)$ from the annual snow crab survey for potential predators of snow crab on the Scotian Shelf: \textbf{Atlantic halibut}.    }
\end{figure}
\clearpage

%--------------------------wolffish as predators------------------------

\begin{figure}
\centering
\subfloat{\includegraphics[width=0.49\textwidth]{\h \es \mb ms\D no\D 50/ms\D no\D 50\D \yrminusthree.png}}\ 
\subfloat{\includegraphics[width=0.49\textwidth]{\h \es \mb ms\D no\D 50/ms\D no\D 50\D \yrminustwo.png}}\\ 
\subfloat{\includegraphics[width=0.49\textwidth]{\h \es \mb ms\D no\D 50/ms\D no\D 50\D \yrminusone.png}}\ 
\subfloat{\includegraphics[width=0.49\textwidth]{\h \es \mb ms\D no\D 50/ms\D no\D 50\D \yr.png}}\\ 
\caption{Locations of potential predators of snow crab on the Scotian Shelf: \textbf{Atlantic wolffish}. Scale is $\tfrac{number}{km^2}$.}
\end{figure}


%------------------------time series of wolffish abundance

\clearpage
\begin{figure}
\centering
\includegraphics[width=0.9\textwidth]{\h \es \Ay timeseries/survey/ms\D mass\D 50.pdf}
\caption{Trends in biomass $\left(\tfrac{t}{km^2}\right)$ from the annual snow crab survey for potential predators of snow crab on the Scotian Shelf: \textbf{Atlantic wolffish}.    }
\end{figure}
\clearpage

%--------------------------thorny skate as predators------------------------
\begin{figure}
\centering
\subfloat{\includegraphics[width=0.49\textwidth]{\h \es \mb ms\D no\D 201/ms\D no\D 201\D \yrminusthree.png}}\ 
\subfloat{\includegraphics[width=0.49\textwidth]{\h \es \mb ms\D no\D 201/ms\D no\D 201\D \yrminustwo.png}}\\ 
\subfloat{\includegraphics[width=0.49\textwidth]{\h \es \mb ms\D no\D 201/ms\D no\D 201\D \yrminusone.png}}\ 
\subfloat{\includegraphics[width=0.49\textwidth]{\h \es \mb ms\D no\D 201/ms\D no\D 201\D \yr.png}}\\ 
\caption{Locations of potential predators of snow crab on the Scotian Shelf: \textbf{Thorny Skate}. Scale is $\tfrac{number}{km^2}$.}
\end{figure}


%------------------------time series of thorny skate abundance

\clearpage
\begin{figure}
\centering
\includegraphics[width=0.9\textwidth]{\h \es \Ay timeseries/survey/ms\D mass\D 201.pdf}
\caption{Trends in biomass $\left(\tfrac{t}{km^2}\right)$ from the annual snow crab survey for potential predators of snow crab on the Scotian Shelf: \textbf{Thorny skate}.    }
\end{figure}
\clearpage


%--------------------------smooth skate as predators------------------------
\begin{figure}
\centering
\subfloat{\includegraphics[width=0.49\textwidth]{\h \es \mb ms\D no\D 202/ms\D no\D 202\D \yrminusthree.png}}\ 
\subfloat{\includegraphics[width=0.49\textwidth]{\h \es \mb ms\D no\D 202/ms\D no\D 202\D \yrminustwo.png}}\\ 
\subfloat{\includegraphics[width=0.49\textwidth]{\h \es \mb ms\D no\D 202/ms\D no\D 202\D \yrminusone.png}}\ 
\subfloat{\includegraphics[width=0.49\textwidth]{\h \es \mb ms\D no\D 202/ms\D no\D 202\D \yr.png}}\\ 
\caption{Locations of potential predators of snow crab on the Scotian Shelf: \textbf{Smooth skate}. Scale is $\tfrac{number}{km^2}$.}
\end{figure}


%------------------------time series of smooth skate abundance

\clearpage
\begin{figure}
\centering
\includegraphics[width=0.9\textwidth]{\h \es \Ay /timeseries/survey/ms\D mass\D 202.pdf}
\caption{Trends in biomass $\left(\tfrac{t}{km^2}\right)$ from the annual snow crab survey for potential predators of snow crab on the Scotian Shelf: \textbf{Smooth skate}   }
\end{figure}
\clearpage


%-----------------------------cod as predators------------------------

\begin{figure}
\centering
\subfloat{\includegraphics[width=0.49\textwidth]{\h \es \mb ms\D no\D 10/ms\D no\D 10\D \yrminusthree.png}}\ 
\subfloat{\includegraphics[width=0.49\textwidth]{\h \es \mb ms\D no\D 10/ms\D no\D 10\D \yrminustwo.png}}\\ 
\subfloat{\includegraphics[width=0.49\textwidth]{\h \es \mb ms\D no\D 10/ms\D no\D 10\D \yrminusone.png}}\ 
\subfloat{\includegraphics[width=0.49\textwidth]{\h \es \mb ms\D no\D 10/ms\D no\D 10\D \yr.png}}\\ 
\caption{Locations of potential predators of snow crab on the Scotian Shelf: \textbf{Atlantic cod}. Scale is $\tfrac{number}{km^2}$.}
\end{figure}


%------------------------time series of cod abundance

\clearpage
\begin{figure}
\centering
\includegraphics[width=0.9\textwidth]{\h \es \Ay timeseries/survey/ms\D mass\D 10.pdf}
\caption{Trends in biomass $\left(\tfrac{kg}{km^2}\right)$ from the annual snow crab survey for potential predators of snow crab on the Scotian Shelf: \textbf{Atlantic cod}   }
\end{figure}
\clearpage




%-----------------------------haddock as predators------------------------

\begin{figure}
\centering
\subfloat{\includegraphics[width=0.49\textwidth]{\h \es \mb ms\D no\D 11/ms\D no\D 11\D \yrminusthree.png}}\ 
\subfloat{\includegraphics[width=0.49\textwidth]{\h \es \mb ms\D no\D 11/ms\D no\D 11\D \yrminustwo.png}}\\ 
\subfloat{\includegraphics[width=0.49\textwidth]{\h \es \mb ms\D no\D 11/ms\D no\D 11\D \yrminusone.png}}\ 
\subfloat{\includegraphics[width=0.49\textwidth]{\h \es \mb ms\D no\D 11/ms\D no\D 11\D \yr.png}}\\ 
\caption{Locations of potential predators of snow crab on the Scotian Shelf: \textbf{Atlantic haddock}. Scale is $\tfrac{number}{km^2}$.}
\end{figure}


%------------------------time series of haddock abundance

\clearpage
\begin{figure}
\centering
\includegraphics[width=0.9\textwidth]{\h \es \Ay timeseries/survey/ms\D mass\D 11.pdf}
\caption{Trends in biomass $\left(\tfrac{kg}{km^2}\right)$ from the annual snow crab survey for potential predators of snow crab on the Scotian Shelf: \textbf{Haddock}  }
\end{figure}
\clearpage



%-----------------------------plaice as predators------------------------

\begin{figure}
\centering
\subfloat{\includegraphics[width=0.49\textwidth]{\h \es \mb ms\D no\D 40/ms\D no\D 40\D \yrminusthree.png}}\ 
\subfloat{\includegraphics[width=0.49\textwidth]{\h \es \mb ms\D no\D 40/ms\D no\D 40\D \yrminustwo.png}}\\ 
\subfloat{\includegraphics[width=0.49\textwidth]{\h \es \mb ms\D no\D 40/ms\D no\D 40\D \yrminusone.png}}\ 
\subfloat{\includegraphics[width=0.49\textwidth]{\h \es \mb ms\D no\D 40/ms\D no\D 40\D \yr.png}}\\ 
\caption{Locations of potential predators of snow crab on the Scotian Shelf: \textbf{American Plaice}. Scale is $\tfrac{number}{km^2}$.}
\end{figure}


%------------------------time series of plaice abundance

\clearpage
\begin{figure}
\centering
\includegraphics[width=0.9\textwidth]{\h \es \Ay timeseries/survey/ms\D mass\D 40.pdf}
\caption{Trends in biomass $\left(\tfrac{kg}{km^2}\right)$ from the annual snow crab survey for potential predators of snow crab on the Scotian Shelf: \textbf{American Plaice}  }
\end{figure}
\clearpage


%------------------------time series of grey seal abundance

\clearpage
\begin{figure}
	\centering
	\includegraphics[width=0.95\textwidth]{\h \es \Ay figures/other\D figs/grey_seal_pop.png}
	\caption{Trends in numerical adundance of Northwest Atlantic grey seals.Blue line is 1:1 male:female ratio, red line is 0.69:1. Source: DFO 2017. }
\end{figure}
\clearpage


%-----------------------------shrimp as prey------------------------

\begin{figure}
\centering
\subfloat{\includegraphics[width=0.49\textwidth]{\h \es \mb ms\D no\D 2211/ms\D no\D 2211\D \yrminusthree.png}}\ 
\subfloat{\includegraphics[width=0.49\textwidth]{\h \es \mb ms\D no\D 2211/ms\D no\D 2211\D \yrminusthree.png}}\\ 
\subfloat{\includegraphics[width=0.49\textwidth]{\h \es \mb ms\D no\D 2211/ms\D no\D 2211\D \yrminustwo.png}}\ 
\subfloat{\includegraphics[width=0.49\textwidth]{\h \es \mb ms\D no\D 2211/ms\D no\D 2211\D \yr.png}}\\ 
\caption{Locations of potential prey of snow crab on the Scotian Shelf: \textbf{Northern Shrimp}. Scale is $\tfrac{number}{km^2}$.}
\end{figure}
\clearpage


\begin{figure}
\centering
\includegraphics[width=0.9\textwidth]{\h \es \Ay timeseries/survey/ms\D mass\D 2211.pdf}
\caption{Trends in biomass $\left(\tfrac{kg}{km^2}\right)$ from the annual snow crab survey for potential prey of snow crab on the Scotian Shelf: \textbf{Northern Shrimp}.  }
\end{figure}
\clearpage

%-----------------------------toad crab as comp------------------------

\begin{figure}
\centering
\subfloat{\includegraphics[width=0.49\textwidth]{\h \es \mb ms\D no\D 2521/ms\D no\D 2521\D \yrminusthree.png}}\ 
\subfloat{\includegraphics[width=0.49\textwidth]{\h \es \mb ms\D no\D 2521/ms\D no\D 2521\D \yrminustwo.png}}\\ 
\subfloat{\includegraphics[width=0.49\textwidth]{\h \es \mb ms\D no\D 2521/ms\D no\D 2521\D \yrminusone.png}}\ 
\subfloat{\includegraphics[width=0.49\textwidth]{\h \es \mb ms\D no\D 2521/ms\D no\D 2521\D \yr.png}}\\ 
\caption{Locations of potential competition of snow crab on the Scotian Shelf:  \textbf{Lesser Toad Crab}. Scale is $\tfrac{number}{km^2}$.}
\end{figure}
\clearpage


\begin{figure}
\centering
\includegraphics[width=0.9\textwidth]{\h \es \Ay timeseries/survey/ms\D mass\D 2521.pdf}
\caption{Trends in biomass $\left(\tfrac{kg}{km^2}\right)$ from the annual snow crab survey for potential competition of snow crab on the Scotian Shelf: \textbf{Lesser Toad Crab}.  }
\end{figure}
\clearpage


%-----------------------------jonah crab as comp------------------------

\begin{figure}
\centering
\subfloat{\includegraphics[width=0.49\textwidth]{\h \es \mb ms\D no\D 2511/ms\D no\D 2511\D \yrminusthree.png}}\ 
\subfloat{\includegraphics[width=0.49\textwidth]{\h \es \mb ms\D no\D 2511/ms\D no\D 2511\D \yrminustwo.png}}\\ 
\subfloat{\includegraphics[width=0.49\textwidth]{\h \es \mb ms\D no\D 2511/ms\D no\D 2511\D \yrminusone.png}}\ 
\subfloat{\includegraphics[width=0.49\textwidth]{\h \es \mb ms\D no\D 2511/ms\D no\D 2511\D \yr.png}}\\ 
\caption{Locations of potential competition of snow crab on the Scotian Shelf: \textbf{Jonah Crab}. Scale is $\tfrac{number}{km^2}$}
\end{figure}
\clearpage

%------------------------time series of jonah crab abundance

\begin{figure}
\centering
\includegraphics[width=0.9\textwidth]{\h \es \Ay timeseries/survey/ms\D mass\D 2511.pdf}
\caption{Trends in biomass $\left(\tfrac{kg}{km^2}\right)$ from the annual snow crab survey for potential competition of snow crab on the Scotian Shelf: \textbf{Jonah Crab}.  }
\end{figure}
\clearpage


%------------------BCD infectinons-------------------------------------

\begin{figure}
\centering
\includegraphics[width=1.1\textwidth]{\h \es \Ay figures/other\D figs/six\D year\D bcd\D map.pdf}
\caption{Annual locations of Bitter Crab Disease observations in snow crab trawl survey.}
\end{figure}


%------------------Size frequency of BCD infectinons-------------------------------------

\begin{figure}
\centering
\includegraphics[width=0.9\textwidth]{\h \es \Ay figures/other\D figs/bcd\D lf.pdf}
\caption{Size frequency distribution of snow crab visibly infected with BCD from 2009-present.}
\end{figure}
\clearpage


%------------------Map of Industrial Activity-------------------------------------

\begin{figure}
\centering
\includegraphics[width=\textwidth]{\h \es \Ay figures/other\D figs/CNSOPB_2017_2019.pdf}
\caption{Map of current Canadian Nova Scotia Offshore Petroleum Board call for exploration bids. }
\end{figure}
\clearpage


%------------map of ST Anns Bank MPA

\begin{figure}
	\centering
	\includegraphics[width=\textwidth]{\h \es \A common/st_anns_bank.png}
	\caption{St. Anns Bank marine protected area with sub-zone designations. This area receieved official MPA designation in 2017.}
\end{figure}
\clearpage


%------------spring landings

\begin{figure}
\centering
\includegraphics[width=\textwidth]{\h \es \Ay figures/other\D figs/percent_spring_landings.pdf}
\caption{The percent of total annual snow crab landings caught during the months of April - June separated by crab fishing area.}
\end{figure}
\clearpage


%--------------fishing effort from logbooks------------------------------------

\begin{figure}
\centering
\subfloat{\includegraphics[width=0.49\textwidth,trim={0 1cm 0 1cm}]{\h \es \m logbook/snowcrab/annual/effort/effort\D \yrminusseven.png}}\ 
\subfloat{\includegraphics[width=0.49\textwidth,trim={0 1cm 0 1cm}]{\h \es \m logbook/snowcrab/annual/effort/effort\D \yrminussix.png}}\\ 
\subfloat{\includegraphics[width=0.49\textwidth,trim={0 1cm 0 1cm}]{\h \es \m logbook/snowcrab/annual/effort/effort\D \yrminusfive.png}}\ 
\subfloat{\includegraphics[width=0.49\textwidth,trim={0 1cm 0 1cm}]{\h \es \m logbook/snowcrab/annual/effort/effort\D \yrminusfour.png}}\\ 
\subfloat{\includegraphics[width=0.49\textwidth,trim={0 1cm 0 1cm}]{\h \es \m logbook/snowcrab/annual/effort/effort\D \yrminusthree.png}}\ 
\subfloat{\includegraphics[width=0.49\textwidth,trim={0 1cm 0 1cm}]{\h \es \m logbook/snowcrab/annual/effort/effort\D \yrminustwo.png}}\\
\subfloat{\includegraphics[width=0.49\textwidth,trim={0 1cm 0 1cm}]{\h \es \m logbook/snowcrab/annual/effort/effort\D \yrminusone.png}}\ 
\subfloat{\includegraphics[width=0.49\textwidth,trim={0 1cm 0 1cm}]{\h \es \m logbook/snowcrab/annual/effort/effort\D \yr.png}}\\
\caption{Fishing effort (number of trap hauls/10 x 10km grid) from fisheries logbook data. Note the increase in effort inshore in S-ENS and the almost complete lack of fishing activity in the Glace Bay Hole area (offshore) of N-ENS. For 4X, year refers to the starting year.}
\end{figure}
\clearpage


%-------------number of active vessels

\begin{figure}
\centering
\includegraphics[width=\textwidth]{\h \es \Ay figures/other\D figs/vessels_per_year.pdf}
\caption{Number of active vessels fishing in each of the SSE snow crab fishing areas.  SENS is separated into CFA23 and CFA24 to maintain consistency with historic information. The number of licences within each area has been stable since 2004.  }
\end{figure}
\clearpage


%--------------fishing landings from logbooks------------------------------------

\begin{figure}
\centering
\subfloat{\includegraphics[width=0.49\textwidth,trim={0 1cm 0 1cm}]{\h \es \m logbook/snowcrab/annual/landings/landings\D \yrminusseven.png}}\ 
\subfloat{\includegraphics[width=0.49\textwidth,trim={0 1cm 0 1cm}]{\h \es \m logbook/snowcrab/annual/landings/landings\D \yrminussix.png}}\\ 
\subfloat{\includegraphics[width=0.49\textwidth,trim={0 1cm 0 1cm}]{\h \es \m logbook/snowcrab/annual/landings/landings\D \yrminusfive.png}}\ 
\subfloat{\includegraphics[width=0.49\textwidth,trim={0 1cm 0 1cm}]{\h \es \m logbook/snowcrab/annual/landings/landings\D \yrminusfour.png}}\\
\subfloat{\includegraphics[width=0.49\textwidth,trim={0 1cm 0 1cm}]{\h \es \m logbook/snowcrab/annual/landings/landings\D \yrminusthree.png}}\ 
\subfloat{\includegraphics[width=0.49\textwidth,trim={0 1cm 0 1cm}]{\h \es \m logbook/snowcrab/annual/landings/landings\D \yrminustwo.png}}\\ 
\subfloat{\includegraphics[width=0.49\textwidth,trim={0 1cm 0 1cm}]{\h \es \m logbook/snowcrab/annual/landings/landings\D \yrminusone.png}}\ 
\subfloat{\includegraphics[width=0.49\textwidth,trim={0 1cm 0 1cm}]{\h \es \m logbook/snowcrab/annual/landings/landings\D \yr.png}}\\ 
\caption{Snow crab landings (tons/10 x 10 km grid) from fisheries logbook data. For 4X, year refers to the starting year.}
\end{figure}
\clearpage


%--------------cpue from logbooks------------------------------------

\begin{figure}
\centering
\subfloat{\includegraphics[width=0.49\textwidth,trim={0 1cm 0 1cm}]{\h \es \m logbook/snowcrab/annual/cpue/cpue\D \yrminusseven.png}}\ 
\subfloat{\includegraphics[width=0.49\textwidth,trim={0 1cm 0 1cm}]{\h \es \m logbook/snowcrab/annual/cpue/cpue\D \yrminussix.png}}\\
\subfloat{\includegraphics[width=0.49\textwidth,trim={0 1cm 0 1cm}]{\h \es \m logbook/snowcrab/annual/cpue/cpue\D \yrminusfive.png}}\ 
\subfloat{\includegraphics[width=0.49\textwidth,trim={0 1cm 0 1cm}]{\h \es \m logbook/snowcrab/annual/cpue/cpue\D \yrminusfour.png}}\\
\subfloat{\includegraphics[width=0.49\textwidth,trim={0 1cm 0 1cm}]{\h \es \m logbook/snowcrab/annual/cpue/cpue\D \yrminusthree.png}}\ 
\subfloat{\includegraphics[width=0.49\textwidth,trim={0 1cm 0 1cm}]{\h \es \m logbook/snowcrab/annual/cpue/cpue\D \yrminustwo.png}}\\ 
\subfloat{\includegraphics[width=0.49\textwidth,trim={0 1cm 0 1cm}]{\h \es \m logbook/snowcrab/annual/cpue/cpue\D \yrminusone.png}}\ 
\subfloat{\includegraphics[width=0.49\textwidth,trim={0 1cm 0 1cm}]{\h \es \m logbook/snowcrab/annual/cpue/cpue\D \yr.png}}\\ 
\caption{Catch rates (kg/trap) of snow crab in each 10 x 10km grid from fisheries logbook data.  For 4X, year refers to the starting year.}
\end{figure}
\clearpage


%----------weekly cpue

\begin{figure}
\centering
\includegraphics[width=1.0\textwidth]{\h \es \Ay figures/other\D figs/weekly_cpue_smoothed.pdf}
\caption{Smoothed catch rates (kg / trap haul) by week for the past three seasons. Split season in N-ENS (spring and summer portions) create the apparent gap in N-ENS data within each year.}
\end{figure}
\clearpage


%--------------carapace width from  at sea observers------------------------------------

\begin{figure}
\centering
	\includegraphics[width=1.0\textwidth,trim={0 0 0 1cm },clip]{\h \es\Ay timeseries/observer/cw.pdf}
\caption{Time series of mean carapace width of commercial crab measured by at-sea-observers. For 4X, the year refers to the starting year of the season. }
\end{figure}
\clearpage


%--------------size distribution from at sea observers------------------------------------

\begin{figure}
\centering
	\subfloat{\includegraphics[width=0.32\textwidth]{\h \es \Ay figures/size\D freq/observer/size\D freqcfanorth\yrminusthree.pdf}}\ 
	\subfloat{\includegraphics[width=0.32\textwidth]{\h \es \Ay figures/size\D freq/observer/size\D freqcfasouth\yrminusthree.pdf}}\ 
	\subfloat{\includegraphics[width=0.32\textwidth]{\h \es \Ay figures/size\D freq/observer/size\D freqcfa4x\yrminusthree.pdf}}\\ 
	\subfloat{\includegraphics[width=0.32\textwidth]{\h \es \Ay figures/size\D freq/observer/size\D freqcfanorth\yrminustwo.pdf}}\ 
	\subfloat{\includegraphics[width=0.32\textwidth]{\h \es \Ay figures/size\D freq/observer/size\D freqcfasouth\yrminustwo.pdf}}\
	\subfloat{\includegraphics[width=0.32\textwidth]{\h \es \Ay figures/size\D freq/observer/size\D freqcfa4x\yrminustwo.pdf}}\\ 
	\subfloat{\includegraphics[width=0.32\textwidth]{\h \es \Ay figures/size\D freq/observer/size\D freqcfanorth\yrminusone.pdf}}\
	\subfloat{\includegraphics[width=0.32\textwidth]{\h \es \Ay figures/size\D freq/observer/size\D freqcfasouth\yrminusone.pdf}}\
	\subfloat{\includegraphics[width=0.32\textwidth]{\h \es \Ay figures/size\D freq/observer/size\D freqcfa4x\yrminusone.pdf}}\\
	\subfloat{\includegraphics[width=0.32\textwidth]{\h \es \Ay figures/size\D freq/observer/size\D freqcfanorth\yr.pdf}}\ 
	\subfloat{\includegraphics[width=0.32\textwidth]{\h \es \Ay figures/size\D freq/observer/size\D freqcfasouth\yr.pdf}}\ 
	\subfloat{\includegraphics[width=0.32\textwidth]{\h \es \Ay figures/size\D freq/observer/size\D freqcfa4x\yr.pdf}}\\
\caption{Size frequency distribution of all at-sea-observer monitored snow crab broken down by carapace condition. For 4X, the year refers to the end year of the season. Vertical lines indicate 95 mm CW.}
\end{figure}
\clearpage


%------------------------soft crab by month

\begin{figure}
\centering
	\includegraphics[width=\textwidth]{\h \es \Ay figures/other\D figs/soft_crab_by_month.pdf}
\caption{The percent of sampled snow crab in the soft shelled state (less than 68 durometer) as determined by at-sea-observers from commercial snow crab traps.}
\end{figure}
\clearpage


%--------------------------soft crab maps
% Not included in 2018. BZ

%\begin{figure}
%\centering
%\includegraphics[width=0.32\textwidth]{\h \es assessments/2013/figures/size\D freq/observer/size\D freqcfa4x2013.png}
%\caption{\textbf{PLACEHOLDER} Location of traps sampled by at sea observers with greater than 20\% soft shelled snow crab.}
%\end{figure}
%\clearpage

%-------------------------length frequency histograms for male snow crab

\begin{figure}
\centering
	\includegraphics[width=\textwidth]{\h \es \Ay figures/size\D freq/survey/male.pdf}\\ 
\caption{Size-frequency histograms of carapace width of male snow crabs obtained from the snow crab survey.  }
\end{figure}
\clearpage


%-------------------------length frequency histograms for female snow crab

\begin{figure}
\centering
	\includegraphics[width=\textwidth]{\h \es \Ay figures/size\D freq/survey/female.pdf}\\ 
\caption{ Size-frequency histograms of carapace width of female snow crabs obtained from the snow crab survey. }
\end{figure}
\clearpage

%-------------------------mature sex ratio

\begin{figure}
\centering
	\includegraphics[width=1.0\textwidth]{\h \es \Ay timeseries/survey/sexratio\D mat.pdf}\\ 
\caption{ Annual proportion female of mature snow crab observed in the survey. Since 2001, most of the Scotian Shelf was uniformly male dominated. One standard error bar is presented.}
\end{figure}
\clearpage


%--------------sex ratio maps------------------------------------

\begin{figure}
\centering
	\subfloat{\includegraphics[width=0.49\textwidth,trim={0 1cm 0 1cm}]{\h \es \m survey/snowcrab/annual/sexratio\D mat/sexratio\D mat\D \yrminusseven.png}}\ 
	\subfloat{\includegraphics[width=0.49\textwidth,trim={0 1cm 0 1cm}]{\h \es \m survey/snowcrab/annual/sexratio\D mat/sexratio\D mat\D \yrminussix.png}}\\ 
	\subfloat{\includegraphics[width=0.49\textwidth,trim={0 1cm 0 1cm}]{\h \es \m survey/snowcrab/annual/sexratio\D mat/sexratio\D mat\D \yrminusfive.png}}\
	\subfloat{\includegraphics[width=0.49\textwidth,trim={0 1cm 0 1cm}]{\h \es \m survey/snowcrab/annual/sexratio\D mat/sexratio\D mat\D \yrminusfour.png}}\\ 
	\subfloat{\includegraphics[width=0.49\textwidth,trim={0 1cm 0 1cm}]{\h \es \m survey/snowcrab/annual/sexratio\D mat/sexratio\D mat\D \yrminusthree.png}}\ 
	\subfloat{\includegraphics[width=0.49\textwidth,trim={0 1cm 0 1cm}]{\h \es \m survey/snowcrab/annual/sexratio\D mat/sexratio\D mat\D \yrminustwo.png}}\\
	\subfloat{\includegraphics[width=0.49\textwidth,trim={0 1cm 0 1cm}]{\h \es \m survey/snowcrab/annual/sexratio\D mat/sexratio\D mat\D \yrminusone.png}}\ 
	\subfloat{\includegraphics[width=0.49\textwidth,trim={0 1cm 0 1cm}]{\h \es \m survey/snowcrab/annual/sexratio\D mat/sexratio\D mat\D \yr.png}}\\
\caption{Proportion of females in the mature fraction of the total morphometrically mature segment of snow crabs on the Scotian Shelf with spatial representations generated using thin plate spline interpolations of data from the annual snow crab survey.}
\end{figure}
\clearpage


%-----------sex ratio immature---------------------------

\begin{figure}
\centering
	\includegraphics[width=1.0\textwidth]{\h \es \Ay timeseries/survey/sexratio\D imm.pdf}\\ 
\caption{ Annual sex ratios (proportion female) of immature snow crab on the Scotian Shelf.}
\end{figure}
\clearpage


%-----------sex ratio immature---------------------------

\begin{figure}
\centering
	\subfloat{\includegraphics[width=0.49\textwidth,trim={0 1cm 0 1cm}]{\h \es \m survey/snowcrab/annual/sexratio\D imm/sexratio\D imm\D \yrminusseven.png}}\ 
	\subfloat{\includegraphics[width=0.49\textwidth,trim={0 1cm 0 1cm}]{\h \es \m survey/snowcrab/annual/sexratio\D imm/sexratio\D imm\D \yrminussix.png}}\\ 
	\subfloat{\includegraphics[width=0.49\textwidth,trim={0 1cm 0 1cm}]{\h \es \m survey/snowcrab/annual/sexratio\D imm/sexratio\D imm\D \yrminusfive.png}}\
	\subfloat{\includegraphics[width=0.49\textwidth,trim={0 1cm 0 1cm}]{\h \es \m survey/snowcrab/annual/sexratio\D imm/sexratio\D imm\D \yrminusfour.png}}\\ 
	\subfloat{\includegraphics[width=0.49\textwidth,trim={0 1cm 0 1cm}]{\h \es \m survey/snowcrab/annual/sexratio\D imm/sexratio\D imm\D \yrminusthree.png}}\ 
	\subfloat{\includegraphics[width=0.49\textwidth,trim={0 1cm 0 1cm}]{\h \es \m survey/snowcrab/annual/sexratio\D imm/sexratio\D imm\D \yrminustwo.png}}\\
	\subfloat{\includegraphics[width=0.49\textwidth,trim={0 1cm 0 1cm}]{\h \es \m survey/snowcrab/annual/sexratio\D imm/sexratio\D imm\D \yrminusone.png}}\ 
	\subfloat{\includegraphics[width=0.49\textwidth,trim={0 1cm 0 1cm}]{\h \es \m survey/snowcrab/annual/sexratio\D imm/sexratio\D imm\D \yr.png}}\\
\caption{Morphometrically immature sex ratios (proportion of females in the mature fraction of the total numbers) of snow crabs on the Scotian Shelf with spatial representations generated using thin plate spline interpolations of data from the annual snow crab survey.}
\end{figure}
\clearpage


%-----------females immature---------------------------
\begin{figure}
\centering
	\includegraphics[width=1.0\textwidth]{\h \es \Ay timeseries/survey/totno\D female\D imm.pdf}\\ 
\caption{ Numeric density of immature females in the SSE.}
\end{figure}
\clearpage


%-----------females immature---------------------------

\begin{figure}
\centering
	\subfloat{\includegraphics[width=0.49\textwidth,trim={0 1cm 0 1cm}]{\h \es \m survey/snowcrab/annual/totno\D female\D imm/totno\D female\D imm\D \yrminusseven.png}}\ 
	\subfloat{\includegraphics[width=0.49\textwidth,trim={0 1cm 0 1cm}]{\h \es \m survey/snowcrab/annual/totno\D female\D imm/totno\D female\D imm\D \yrminussix.png}}\\ 
	\subfloat{\includegraphics[width=0.49\textwidth,trim={0 1cm 0 1cm}]{\h \es \m survey/snowcrab/annual/totno\D female\D imm/totno\D female\D imm\D \yrminusfive.png}}\
	\subfloat{\includegraphics[width=0.49\textwidth,trim={0 1cm 0 1cm}]{\h \es \m survey/snowcrab/annual/totno\D female\D imm/totno\D female\D imm\D \yrminusfour.png}}\\ 
	\subfloat{\includegraphics[width=0.49\textwidth,trim={0 1cm 0 1cm}]{\h \es \m survey/snowcrab/annual/totno\D female\D imm/totno\D female\D imm\D \yrminusthree.png}}\ 
	\subfloat{\includegraphics[width=0.49\textwidth,trim={0 1cm 0 1cm}]{\h \es \m survey/snowcrab/annual/totno\D female\D imm/totno\D female\D imm\D \yrminustwo.png}}\\
	\subfloat{\includegraphics[width=0.49\textwidth,trim={0 1cm 0 1cm}]{\h \es \m survey/snowcrab/annual/totno\D female\D imm/totno\D female\D imm\D \yrminusone.png}}\ 
	\subfloat{\includegraphics[width=0.49\textwidth,trim={0 1cm 0 1cm}]{\h \es \m survey/snowcrab/annual/totno\D female\D imm/totno\D female\D imm\D \yr.png}}\\
\caption{Numerical densities $\left(\tfrac{number}{km^2}\right)$ of the immature female snow crabs on the Scotian Shelf with spatial representation generated using using thin plate spline interpolations of data from the annual snow crab survey. }
\end{figure}
\clearpage


%-----------females mature---------------------------

\begin{figure}
\centering
	\includegraphics[width=1.0\textwidth]{\h \es \Ay timeseries/survey/totno\D female\D mat.pdf}\\ 
\caption{ Numeric density of mature females from the annual snow crab survey.}
\end{figure}
\clearpage


%-----------females mature---------------------------

\begin{figure}
\centering
	\subfloat{\includegraphics[width=0.49\textwidth,trim={0 1cm 0 1cm}]{\h \es \m survey/snowcrab/annual/totno\D female\D mat/totno\D female\D mat\D \yrminusseven.png}}\ 
	\subfloat{\includegraphics[width=0.49\textwidth,trim={0 1cm 0 1cm}]{\h \es \m survey/snowcrab/annual/totno\D female\D mat/totno\D female\D mat\D \yrminussix.png}}\\ 
	\subfloat{\includegraphics[width=0.49\textwidth,trim={0 1cm 0 1cm}]{\h \es \m survey/snowcrab/annual/totno\D female\D mat/totno\D female\D mat\D \yrminusfive.png}}\
	\subfloat{\includegraphics[width=0.49\textwidth,trim={0 1cm 0 1cm}]{\h \es \m survey/snowcrab/annual/totno\D female\D mat/totno\D female\D mat\D \yrminusfour.png}}\\ 
	\subfloat{\includegraphics[width=0.49\textwidth,trim={0 1cm 0 1cm}]{\h \es \m survey/snowcrab/annual/totno\D female\D mat/totno\D female\D mat\D \yrminusthree.png}}\ 
	\subfloat{\includegraphics[width=0.49\textwidth,trim={0 1cm 0 1cm}]{\h \es \m survey/snowcrab/annual/totno\D female\D mat/totno\D female\D mat\D \yrminustwo.png}}\\
	\subfloat{\includegraphics[width=0.49\textwidth,trim={0 1cm 0 1cm}]{\h \es \m survey/snowcrab/annual/totno\D female\D mat/totno\D female\D mat\D \yrminusone.png}}\ 
	\subfloat{\includegraphics[width=0.49\textwidth,trim={0 1cm 0 1cm}]{\h \es \m survey/snowcrab/annual/totno\D female\D mat/totno\D female\D mat\D \yr.png}}\\
\caption{Numerical densities $\left(\tfrac{number}{km^2}\right)$ of the mature female snow crabs on the Scotian Shelf with spatial representation generated using thin plate spline interpolations of data from the annual snow crab survey. }
\end{figure}
\clearpage




%-----------females berried---------------------------

\begin{figure}
\centering
	\subfloat{\includegraphics[width=0.49\textwidth,trim={0 1cm 0 1cm}]{\h \es \m survey/snowcrab/annual/totno\D female\D berried/totno\D female\D berried\D \yrminusseven.png}}\ 
	\subfloat{\includegraphics[width=0.49\textwidth,trim={0 1cm 0 1cm}]{\h \es \m survey/snowcrab/annual/totno\D female\D berried/totno\D female\D berried\D \yrminussix.png}}\\ 
	\subfloat{\includegraphics[width=0.49\textwidth,trim={0 1cm 0 1cm}]{\h \es \m survey/snowcrab/annual/totno\D female\D berried/totno\D female\D berried\D \yrminusfive.png}}\
	\subfloat{\includegraphics[width=0.49\textwidth,trim={0 1cm 0 1cm}]{\h \es \m survey/snowcrab/annual/totno\D female\D berried/totno\D female\D berried\D \yrminusfour.png}}\\ 
	\subfloat{\includegraphics[width=0.49\textwidth,trim={0 1cm 0 1cm}]{\h \es \m survey/snowcrab/annual/totno\D female\D berried/totno\D female\D berried\D \yrminusthree.png}}\ 
	\subfloat{\includegraphics[width=0.49\textwidth,trim={0 1cm 0 1cm}]{\h \es \m survey/snowcrab/annual/totno\D female\D berried/totno\D female\D berried\D \yrminustwo.png}}\\
	\subfloat{\includegraphics[width=0.49\textwidth,trim={0 1cm 0 1cm}]{\h \es \m survey/snowcrab/annual/totno\D female\D berried/totno\D female\D berried\D \yrminusone.png}}\ 
	\subfloat{\includegraphics[width=0.49\textwidth,trim={0 1cm 0 1cm}]{\h \es \m survey/snowcrab/annual/totno\D female\D berried/totno\D female\D berried\D \yr.png}}\\
\caption{Numerical densities $\left(\tfrac{number}{km^2}\right)$ of the berried female snow crabs on the Scotian Shelf with spatial representation generated using thin plate spline interpolations of data from the annual snow crab survey.}
\end{figure}
\clearpage


%-----------egg production---------------------------

\begin{figure} %
\centering
	\includegraphics[width=1.0\textwidth]{\h \es \Ay timeseries/survey/fecundity.pdf}\\ 
\caption{Index of egg production in the SSE, determined from the number of berried females and fecundity at weight estimates.}
\end{figure}
\clearpage


%-----------time series of fishable biomass---------------------------

\begin{figure}
\centering
	\includegraphics[width=1.0\textwidth]{\h \es \Ay timeseries/survey/R0\D mass.pdf}\\ 
\caption{Trends in the geometric mean of fishable biomass $\left(\tfrac{t}{km^2}\right)$ obtained from the annual snow crab survey. Error bars are 95\% CI about geometric mean.}
\end{figure} %
\clearpage


%-----------area expanded time series of fishable biomass---------------------------

\begin{figure}
\centering
\includegraphics[width=\textwidth]{\h \es \Ay timeseries/interpolated/R0\D mass.png}\\ 
\caption{Trends in the area expanded geometric mean fishable biomass $\left(\tfrac{t}{km^2}\right)$ obtained from the annual snow crab survey. Error bars are 95\% CI about geometric mean. Area estimates are obtained from \textbf{stmv}. Vertical dashed line represents timing shift from a spring survey to a fall survey. Horizontal dashed line is mean.}
\end{figure} 
\clearpage


%-----------fishable biomass---------------------------

\begin{figure}
\centering
	\subfloat{\includegraphics[width=0.49\textwidth,trim={0 1cm 0 1cm}]{\h \es \m survey/snowcrab/annual/R0\D mass/R0\D mass\D \yrminusseven.png}}\ 
	\subfloat{\includegraphics[width=0.49\textwidth,trim={0 1cm 0 1cm}]{\h \es \m survey/snowcrab/annual/R0\D mass/R0\D mass\D \yrminussix.png}}\\ 
	\subfloat{\includegraphics[width=0.49\textwidth,trim={0 1cm 0 1cm}]{\h \es \m survey/snowcrab/annual/R0\D mass/R0\D mass\D \yrminusfive.png}}\
	\subfloat{\includegraphics[width=0.49\textwidth,trim={0 1cm 0 1cm}]{\h \es \m survey/snowcrab/annual/R0\D mass/R0\D mass\D \yrminusfour.png}}\\ 
	\subfloat{\includegraphics[width=0.49\textwidth,trim={0 1cm 0 1cm}]{\h \es \m survey/snowcrab/annual/R0\D mass/R0\D mass\D \yrminusthree.png}}\ 
	\subfloat{\includegraphics[width=0.49\textwidth,trim={0 1cm 0 1cm}]{\h \es \m survey/snowcrab/annual/R0\D mass/R0\D mass\D \yrminustwo.png}}\\
	\subfloat{\includegraphics[width=0.49\textwidth,trim={0 1cm 0 1cm}]{\h \es \m survey/snowcrab/annual/R0\D mass/R0\D mass\D \yrminusone.png}}\ 
	\subfloat{\includegraphics[width=0.49\textwidth,trim={0 1cm 0 1cm}]{\h \es \m survey/snowcrab/annual/R0\D mass/R0\D mass\D \yr.png}}\\
\caption{Fishable biomass densities $\left(\tfrac{t}{km^2}\right)$ on the SSE with spatial representation generated using thin plate spline interpolations of data from the annual snow crab survey. }
\end{figure}
\clearpage


%-----------recruits---------------------------

\begin{figure}
\centering
	\includegraphics[width=1.0\textwidth]{\h \es \Ay timeseries/survey/R2\D no.pdf}\\ 
\caption{Trends in the geometric mean abundance of male snowcrab (75-95 mm CW) obtained from the annual snow crab survey. Error bars are 95\% CI about geometric mean.}
\end{figure}
\clearpage


%--------------priors and posteriors


\begin{figure}   
    \includegraphics[width=\textwidth]{\h \es \Ay r\D density.png}
\caption{Prior (red) and posterior (bars) distribution for populaton growth parameter, r, from the biomass dynamic model of snow crab production in crab fishing areas on the Scotian Shelf. Within each panel, estimates of posterior median and 95\% credible intervals are given in the legend.}
\end{figure}
\clearpage

\begin{figure} 
	\includegraphics[width=\textwidth]{\h \es \Ay K\D density.png}
	\caption{Prior (red) and posterior (bars) distribution for carrying capacity parameter, K, from the biomass dynamic model of snow crab production in crab fishing areas on the Scotian Shelf. Within each panel, estimates of posterior median and 95\% credible intervals are given in the legend.} 
\end{figure}
\clearpage


\begin{figure}  
    \includegraphics[width=\textwidth]{\h \es \Ay q\D density.png}
\caption{Prior (red) and posterior (bars) distribution for catchability parameter, q, from the biomass dynamic model of snow crab production in crab fishing areas on the Scotian Shelf. Within each panel, estimates of posterior median and 95\% credible intervals are given in the legend.}  
\end{figure}
\clearpage


\begin{figure}    
    \includegraphics[width=\textwidth]{\h \es \Ay bpsd\D density.png}
\caption{Prior (red) and posterior (bars) distribution for process error from the biomass dynamic model of snow crab production in crab fishing areas on the Scotian Shelf. Within each panel, estimates of posterior median and 95\% credible intervals are given in the legend.}  
\end{figure}
\clearpage


\begin{figure}    
    \includegraphics[width=\textwidth]{\h \es \Ay bosd\D density.png}
   \caption{Prior (red) and posterior (bars) distribution for observation error from the biomass dynamic model of snow crab production in crab fishing areas on the Scotian Shelf. Within each panel, estimates of posterior median and 95\% credible intervals are given in the legend.} 
\end{figure}
\clearpage



\begin{figure}   
    \includegraphics[width=\textwidth]{\h \es \Ay FMSY\D density.png}
\caption{Posterior distribution for fishing mortality at maximum sustainable yield from the biomass dynamic model of snow crab production in crab fishing areas on the Scotian Shelf. Within each panel, estimates of posterior median and 95\% credible intervals are given in the legend.}  
\end{figure}
\clearpage

%-----------Modelsimulations of fishable biomass---------------------------

\begin{figure}
	\centering
	\includegraphics[width=\textwidth]{\h \es \Ay biomass\D timeseries.png}\\ 
	\caption{Time series of fishable biomass from the logistic population models. The fishable biomass index is shown in red dashed lines. The q-corrected fishable biomass index is shown in green dashed lines. The posterior mean fishable biomass estimated from the logistic model are shown in blue stippled lines. The density distribution of posterior fishable biomass estimates are presented with 95\% CI (gray) with the darkest area being medians. }
\end{figure}
\clearpage


%-----------Modelsimulations of fishing mortality---------------------------

\begin{figure}
\centering
	\includegraphics[width=\textwidth]{\h \es \Ay fishingmortality\D timeseries.png}\\
\caption{Time-series of fishing mortality from the logistic population models for N-ENS, S-ENS and 4X, respectively. Posterior density distributions are presented in gray, with the darkest line being the median with 95\% CI. The red line is the estimated FMSY and dark stippled line is the 20\% harvest rate.}
\end{figure}
\clearpage


%---------Harvest Control Rule--------------------------------------

\begin{figure}
\centering
	\includegraphics[width=\textwidth]{\h \es \A common/hcr_ideal.pdf}\\ 
\caption{Harvest control rules for the SSE snow crab fishery.}
\end{figure}
\clearpage

%---------relationship between F and B--------------------------------------

\begin{figure}
\centering
	\includegraphics[width=\textwidth]{\h \es \Ay hcr\D default.png}\\ 
\caption{Time series of fishing mortality and pre-fishery biomass for N-ENS (top), S-ENS (middle) and 4X (bottom) as obtained from the logistic population models.}
\end{figure}
\clearpage


%
%---------relationship between Catch and B--------------------------------------
%Not included in 2017 and beyond. BZ
%\begin{figure}
%\centering
%\includegraphics[width=\textwidth]{\h \es assessments/2014/figures/bugs/survey/hcr\D simple.png}\\ 
%\caption{Fishery catch as a function of fishable biomass for N-ENS (top), S-ENS (middle) and 4X (bottom). Exploitation rates of 20\% are %indicated by the solid gray line. Bounding this are the lines associated with 10\% and 30\% exploitation rates, in dashed lines}
%\end{figure}
%\clearpage


\end{document}
