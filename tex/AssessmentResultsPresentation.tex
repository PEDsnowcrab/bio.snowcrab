
\newcommand{\D}{.}
%{\includegraphics[width=\textwidth]{\es \A common/map\D scotianshelf.png}}{Map of Scotian Shelf}


% Sample Block- text box with figure
%%------------------------------------------------
%\begin{frame}
%\frametitle{Year 1- July 2013}

%\begin{block}

%\begin{itemize}
%\item Acoustic tag  27 crab (tags emit signal \textgreater 3 years)
%\begin{itemize}
	%\item Last 1-2 Days of Inner NENS July Fishery
	%\item Release $ \sim4 $  km on either side of proposed line
	%\item Actively track movement through time.
	%\begin{itemize}
	%\item Grid search
	%\end{itemize}
	%\item Passive tracking 
	%\begin{itemize}
	%\item Cabot strait line
	%\item Wave glider
	%\end{itemize}

%\end{itemize}
%\end{itemize}
%\end{block}
%\begin{figure}
%\vspace*{-0.5cm}
%\centerline{\includegraphics[width=0.4\textwidth]{Listeninggrid.pdf}}
%\end{figure}

%\end{frame}


%%---------

%----------------------------------------------------------------------------------------
%	PACKAGES AND THEMES
%----------------------------------------------------------------------------------------

\documentclass{beamer}

\mode<presentation> {

\usetheme{Hannover}
\usetheme{Boadilla}

%\usetheme{AnnArbor}

%\usecolortheme{dolphin}
\usecolortheme{seagull}

%\usefonttheme{structuresmallcapsserif}

}

\usepackage{graphicx} % Allows including images
\usepackage{graphics}
\usepackage{booktabs} % Allows the use of \toprule, \midrule and \bottomrule in tables

\newcommand{\e}{/home/michelle/bio.data/}
\newcommand{\es}{snowcrab/}
\newcommand{\Ay}{assessments/2015/}
\newcommand{\A}{assessments/}

%----------------------------------------------------------------------------------------
%	TITLE PAGE
%----------------------------------------------------------------------------------------

\title{2015 4VWX Snow Crab Assessment Results} % The short title appears at the bottom of every slide, the full title is only on the title page

\author{Snow Crab Unit} % Your name
\institute[DFO] % Your institution as it will appear on the bottom of every slide, may be shorthand to save space
{
Department of Fisheries and Oceans \\ % Your institution for the title page
\medskip
\textit{} % Your email address
}
\date{\the\year} % Date, can be changed to a custom date

\begin{document}

\begin{frame}
\titlepage % Print the title page as the first slide
\end{frame}

\begin{frame}
\frametitle{Overview} % Table of contents slide, comment this block out to remove it
\tableofcontents % Throughout your presentation, if you choose to use \section{} and \subsection{} commands, these will automatically be printed on this slide as an overview of your presentation
\end{frame}

%----------------------------------------------------------------------------------------
%	PRESENTATION SLIDES
%----------------------------------------------------------------------------------------



\begin{frame}
\frametitle{4VWX Snow Crab}
	\begin{columns}[T]
	\begin{column}{0.4\textwidth}
	\begin{itemize}
		\item Southern limit of Snow Crab extent
		\item Three commercial fishing areas
	\end{itemize}
	\end{column}
	
	\begin{column}{0.6\textwidth}
	\begin{centering}
	Scotian Shelf and CFAs
		\begin{figure}
  		\includegraphics[width=\textwidth]{\e \es maps/Basemap.pdf}
  		\end{figure}
  		\end{centering}
 	\end{column}
 	\end{columns}
\end{frame}

%--------------------------------------------------------------------------------

\section{Survey}

\begin{frame}
\frametitle{Survey 2015}
	\begin{columns}
	\begin{column}{0.3\textwidth}
	Survey:
	\begin{itemize}
	\begin{tiny}
		\item Western component of 4x completed this year
		\item More fishing days than normal due to poor weather conditions
		\item Same vessel, captain and Science staff as 2014\\
	\end{tiny}
	\end{itemize}

	Area Patterns:
	\begin{itemize}
	\begin{tiny}
		\item N-ENS: Less crab pretty much everywhere in 2015, except inner Glace Bay Hole
		\item S-ENS: Less crab inshore and on the slope edge in 2015
		\item 4x: Slightly more in some locations, less in others. Despite suspect numbers from last year\\
	\end{tiny}
	\end{itemize}
	\end{column}
	
	\begin{column}{0.7\textwidth}
		\begin{figure}
  		\includegraphics[width=\textwidth]{\e \es \Ay figures/Basemap_yearlydifference.pdf}
  		\end{figure}
 	\end{column}
 	\end{columns}
\end{frame}
%--------------------------------------------------------------------------------

\section{Interactions with Other Species}

\begin{frame}
\frametitle{Interactions with Other Species}

\begin{itemize}
	\item Competitors: Competition with some benthic fish and crabs, but few strong competitors
	\item Prey: Echinoderms, shrimp, crabs, worms, bivalves, sea stars
	\item Predators: Atlantic Halibut, Atlantic Wolfish, Skates, Longhorn Cculpin, Sea Raven, Atlantic Cod, White Hake, American Plaice, and Haddock
\end{itemize}
\end{frame}

%------------------------------------------------------------------------------
\subsection{Competition/ Prey}

\begin{frame}
\frametitle{Competition/Prey}
Trends in biomass for potential predators/prey (log10, no\textbackslash$km^2$) of Snow Crab on the Scotian Shelf, in the Snow Crab Survey
	\begin{columns}
	\begin{column}{0.25\textwidth}
	\begin{itemize}
	  \setlength\itemsep{2em}
		\item[] Jonah Crab 
		\item[] Lesser Toad Crab
		\item[] Northern Shrimp
	\end{itemize}
	\end{column}

	\begin{column}{0.25\textwidth}
 	\begin{figure}
    \includegraphics[width=\textwidth]{\e \es R/maps/species/snowcrab/jonahcrab/annual/totno/totno\D 2014.png}\\   
    \includegraphics[width=\textwidth]{\e \es R/maps/species/snowcrab/lessertoadcrab/annual/totno/totno\D 2014.png}\\   
    \includegraphics[width=\textwidth]{\e \es R/maps/species/snowcrab/northernshrimp/annual/totno/totno\D 2014.png}  
  	\end{figure}
  	\end{column}

  	\begin{column}{0.25\textwidth}
 	\begin{figure}
    \includegraphics[width=\textwidth]{\e \es R/maps/species/snowcrab/jonahcrab/annual/totno/totno\D 2015.png}\\   
    \includegraphics[width=\textwidth]{\e \es R/maps/species/snowcrab/lessertoadcrab/annual/totno/totno\D 2015.png}\\   
    \includegraphics[width=\textwidth]{\e \es R/maps/species/snowcrab/northernshrimp/annual/totno/totno\D 2015.png}  
  	\end{figure}
  	\end{column}

	\begin{column}{0.25\textwidth}
 	\begin{figure}
  	\includegraphics[width=0.6\textwidth]{\e \es \Ay timeseries/survey/ms\D mass\D 2511.pdf}\\   
    \includegraphics[width=0.6\textwidth]{\e \es \Ay timeseries/survey/ms\D mass\D 2521.pdf}\\
    \includegraphics[width=0.6\textwidth]{\e \es \Ay timeseries/survey/ms\D mass\D 2211.pdf}
	\end{figure}
  	\end{column}

  	\end{columns}
\end{frame}

%------------------------------------------------
\subsection{Predation}

\begin{frame}
\frametitle{Potential Predation}
Trends in biomass for potential predators (log10, no\textbackslash$km^2$) of Snow Crab on the Scotian Shelf, in the Snow Crab Survey
	\begin{columns}
	\begin{column}{0.25\textwidth}
	\begin{itemize}
	  \setlength\itemsep{2em}
		\item[] Halibut 
		\item[] Atlantic Cod 
		\item[] Thorny Skate 
	\end{itemize}
	\end{column}

	\begin{column}{0.25\textwidth}
 	\begin{figure}
    \includegraphics[width=\textwidth]{\e \es R/maps/species/snowcrab/halibut/annual/totno/totno\D 2014.png}\\   
    \includegraphics[width=\textwidth]{\e \es R/maps/species/snowcrab/cod/annual/totno/totno\D 2014.png}\\   
    \includegraphics[width=\textwidth]{\e \es R/maps/species/snowcrab/thornyskate/annual/totno/totno\D 2014.png}  
  	\end{figure}
  	\end{column}

  	\begin{column}{0.25\textwidth}
 	\begin{figure}
    \includegraphics[width=\textwidth]{\e \es R/maps/species/snowcrab/halibut/annual/totno/totno\D 2015.png}\\   
    \includegraphics[width=\textwidth]{\e \es R/maps/species/snowcrab/cod/annual/totno/totno\D 2015.png}\\   
    \includegraphics[width=\textwidth]{\e \es R/maps/species/snowcrab/thornyskate/annual/totno/totno\D 2015.png}  
  	\end{figure}
  	\end{column}

	\begin{column}{0.25\textwidth}
 	\begin{figure}
  	\includegraphics[width=0.6\textwidth]{\e \es \Ay timeseries/survey/ms\D mass\D 30.pdf}\\   
    \includegraphics[width=0.6\textwidth]{\e \es \Ay timeseries/survey/ms\D mass\D 10.pdf}\\
    \includegraphics[width=0.6\textwidth]{\e \es \Ay timeseries/survey/ms\D mass\D 201.pdf}
	\end{figure}
  	\end{column}

  	\end{columns}
\end{frame}
%------------------------------------------------

\section{Population Assessment}
\subsection{Recruitment}

\begin{frame}
\frametitle{Recruitment}
\begin{columns}
	\begin{column}{0.5\textwidth}
	Size-frequency histograms of carapace width of male Snow Crab. The vertical line represents the legal size (95 mm)
	\begin{itemize}
		\item S-ENS: Stable recruitment
		\item N-ENS: A gap in recruitment, should enter the fishery in 2-3 years
		\item 4x: Minimal internal recruitment for the forseeable future  
	\end{itemize}
	\end{column}

	\begin{column}{0.5\textwidth}
	\begin{figure}
		\includegraphics[width=\textwidth]{\e \es \Ay figures/size\D freq/survey/male.pdf}
	\end{figure}
	\end{column}
	\end{columns}
\end{frame}

%--------------------------------------------------------------
\subsection{Reproductive Potential}

\begin{frame}
\frametitle{Sex Ratios}
	\begin{center}
	Proportion of females in the mature population
	\end{center}
	
	\begin{columns}
	\begin{column}{0.3\textwidth}
	\begin{itemize}
	\begin{footnotesize}
		\item S-ENS: Stable, much higher inshore this year...highest since 2008
		\item N-ENS: Recovering from near zero
		\item 4x: Continued high proportion of males  
	\end{footnotesize}
	\end{itemize}
	\end{column}

	\begin{column}{0.35\textwidth}
	\begin{figure}
	\includegraphics[width=\textwidth]{\e \es R/maps/survey/snowcrab/annual/sexratio\D mat/sexratio\D mat\D 2015.png}
	\end{figure}
	\end{column}

	\begin{column}{0.35\textwidth}
	\begin{figure}
	 \includegraphics[width=\textwidth]{\e \es \Ay timeseries/survey/sexratio\D mat.pdf}
	\end{figure}
	\end{column}
	\end{columns}
\end{frame}

%------------------------------------------------
\begin{frame}
\frametitle{Female Size-Frequency}
\begin{columns}
	\begin{column}{0.5\textwidth}
		\begin{itemize}
		\item Size-frequency histograms of carapace width of female Snow Crab
		\item Newly matured female crab are expected in all areas for the next 3-4 years
		\item Each newly matured female should support egg production for 3-5 years
		\end{itemize}
	\end{column}

	\begin{column}{0.5\textwidth}
	\begin{figure}
		\includegraphics[width=\textwidth]{\e \es \Ay figures/size\D freq/survey/female.pdf}
	\end{figure}
	\end{column}
	\end{columns}
\end{frame}

%------------------------------------------------
\begin{frame}
\frametitle{Primiparous/Multiparous}
\begin{columns}
	\begin{column}{0.5\textwidth}
	\begin{center}
	Primiparous - first time birth
	\end{center}
		\begin{figure}
		\includegraphics[width=\textwidth]{\e \es R/maps/survey/snowcrab/annual/totno\D female\D primiparous/totno\D female\D primiparous\D 2015.png}
	\end{figure}
	\end{column}

	\begin{column}{0.5\textwidth}
	\begin{center}
	Multiparous - given birth 2+ times
	\end{center}
	\begin{figure}
		\includegraphics[width=\textwidth]{\e \es R/maps/survey/snowcrab/annual/totno\D female\D multiparous/totno\D female\D multiparous\D 2015.png}	
	\end{figure}
	\end{column}
	\end{columns}
\end{frame}

%------------------------------------------------
\begin{frame}
\frametitle{Mature Females}
	\begin{center}
	Total number of females in the mature population. Reproductive potential peaked in 2007/8 and continues to remain low (except 4x in 2012).
	\end{center}
	
	\begin{columns}
	\begin{column}{0.5\textwidth}
	\begin{figure}
	  \includegraphics[width=0.7\textwidth]{\e \es \Ay timeseries/survey/totno\D female\D mat\D combined.pdf}
	\end{figure}
	\end{column}

	\begin{column}{0.5\textwidth}
	\begin{figure}
	\includegraphics[width=\textwidth]{\e \es R/maps/survey/snowcrab/annual/totno\D female\D mat/totno\D female\D mat\D 2015.png}	
	\end{figure}
	\end{column}
	\end{columns}
\end{frame}

%------------------------------------------------
\subsection{Fishable Biomasss}

\begin{frame}
\frametitle{Fishable Biomass Index}
\begin{columns}

\begin{column}{0.5\textwidth}
	\begin{center}
	Fishable Biomass from Annual Snow Crab Survey. Log10 (t\textbackslash$km^2$).\\ 
	\end{center}
	\begin{figure}
		\includegraphics[width=0.9\textwidth]{\e \es R/maps/survey/snowcrab/annual/R0\D mass/R0\D mass\D 2015.png}
	\end{figure}
\end{column}

\begin{column}{0.5\textwidth}
	\begin{center}
	Time series of the Fishable Biomass from Annual Snow Crab Survey
	\end{center}
\begin{figure}
    \centering
    \includegraphics[width=0.7\textwidth]{\e \es \Ay timeseries/survey/R0\D mass.pdf}
 \end{figure}
\end{column}

\end{columns}
\end{frame}


%------------------------------------------------

\begin{frame}
\frametitle{Area Expanded\
Fishable Biomass Index}
\begin{columns}

\begin{column}{0.5\textwidth}
	\begin{center}
	Time series of the Area Expanded Fishable Biomass Density from Annual Snow Crab Survey. Area estimates obtained from the GAM habitat model.
	\end{center}
%\begin{figure}
 %   \centering
  %  \includegraphics[width=0.7\textwidth]{\e \es \Ay timeseries/survey/R0\D mass.pdf}
 %\end{figure}
\end{column}

\begin{column}{0.5\textwidth}
	\begin{center}
	Potential Habitat Map
	\end{center}
%	\begin{figure}

%	\end{figure}
	\end{column}
\end{columns}
\end{frame}


%------------------------------------------------
\section{Fisheries Model}
\subsection{Fishable Biomass}

\begin{frame}
\frametitle{Fishable Biomass}
Time series of fishable biomass from the logistic population model. A three year projection assuming a constant exploitation strategy of 20\% is also provided.
\begin{columns}
\begin{column}{0.5\textwidth}
\begin{itemize}
\begin{footnotesize}
	\item Red Dashed Lines: The fishable biomass index
	\item Blue Dashed Lines: The posterior mean fishable biomass estimated from the logistic model
	\item Grey Area: The density distribution of posterior fishable biomass estimates darkest area being medians and the 95\% Credible Intervals (CI). 
\end{footnotesize}
\end{itemize}
\end{column}

\begin{column}{0.5\textwidth}
\begin{figure}[ht]
    \centering
    \includegraphics[width=0.9\textwidth]{\e \es \Ay figures/bugs/survey/biomass\D timeseries.png}
\end{figure}
\end{column}

\end{columns}
\end{frame}

%------------------------------------------------
\subsection{Fishing Mortality}
\begin{frame}
\frametitle{Fishing Mortality}
\begin{columns}
\begin{column}{0.4\textwidth}
Mortality factors are associated with:
\begin{itemize}
\begin{footnotesize}
	\item Predation: Long-term risk if increase in predators
	\item Food limitation: Likely stable 
	\item Competition: Likely stable\\
\end{footnotesize}
\end{itemize}

Area trends:
\begin{itemize}
\begin{footnotesize}
	\item N-ENS: Increasing over the past several years
	\item S-ENS: Slightly above 20\% harvest rate
	\item 4x: Peaking in 2005 and 2011/2012
\end{footnotesize}
\end{itemize}
\end{column}

\begin{column}{0.6\textwidth}
\begin{footnotesize}
Time series of fishing mortality from the logistic population model 
\end{footnotesize}

\begin{itemize}
\begin{tiny}
	\item[] Grey Area: Posterior density distributions, with median with 90\% CI 
	\item[] Dark-dashed line: is the 20\% harvest rate
	\item[] Red line: estimated FMSY 	
\end{tiny}
\end{itemize}

\begin{figure}
    \includegraphics[width=0.6\textwidth]{\e \es \Ay figures/bugs/survey/fishingmortality\D timeseries.png}
\end{figure}
\end{column}

\end{columns}
\end{frame}


%------------------------------------------------
\section{Business}
\subsection{Meetings / Document Schedule}

\begin{frame}

\frametitle{Meetings Schedule}

 \vspace*{-0.5cm}
\begin{block}

\begin{itemize}
\item Past- Pre-Rap, RAP, Advisory Committee (AC) Meetings annually
\item New DFO standard- Framework, RAP every five years, AC Annually
\item Snow Crab Plan- Pre-Rap, Informal RAP, AC annually, Framework/Formal RAP every ~5 years

\end{itemize}
\end{block}
\end{frame}

%----------------------------------------------------------------------------------------


\begin{frame}

\frametitle{Document Schedule}

 \vspace*{-0.5cm}
\begin{block}

\begin{itemize}
\item Past- Res Doc (100+ pages) and SAR (10-15 pages) annually
\item New DFO standard- Update document (2-3 pages) annually, Res Doc every 5 years
\item Snow Crab Plan- Update document (10-15 pages) annually, based on SAR, Res Doc every 5 years
\item Full assessment run annually, an items of note included in update document

\end{itemize}

\end{block}

\end{frame}
%------------------------------------------------



\end{document}

